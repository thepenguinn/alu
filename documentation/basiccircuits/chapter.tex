\documentclass[../main]{subfiles}

\ifSubfilesClassLoaded{
    %% % \tikzset{clkGen/.style = {muxdemux, muxdemux def = {
%     NL = 3, NR = 4, NB = 0, NT = 0,
%     Lh = 3, Rh = 3,
%     w = 5 },
%     draw only left pins = {1} }
% }

\tikzset{
    muxtest/.style = {
        muxdemux, muxdemux def={NL=16, NR=3, NT=5, NB=3, w=2,
            inset w=0.5, Lh=4, inset Lh=2.0, inset Rh=1.0,
            square pins=1},
        draw only right pins={1,3},
        draw only top pins={1-3},
        draw only bottom pins={3}
    }
}

\tikzset{mux16/.style = {muxdemux, muxdemux def = {
    NL = 21, NR = 3, NB = 6, NT = 0,
    Lh = 42, Rh = 27,
    inset Lh = 0, inset Rh = 0,
    inset w = 0,
    %% inset Lh = 5, inset Rh = 2.75,
    %% inset w = 1.25,
    w = 9,
    square pins = 1
    },
    draw only left pins={2-5,7-10,12-15,17-20},
    draw only right pins={2},
    draw only bottom pins={2-5}
    }
}

    %% \ctikzsubcircuitdef{subctkmuxFirstLayer} {
    in0, in1, out0, out1,
    lsl0, lsl1, lsl2, lsl3,
    rsl0, rsl1, rsl2, rsl3,
    blsl0, blsl1, blsl2, blsl3,
    bin0, bin1, bout0, bout1%
} {
    node (#1) [muxFirstLayer]{}%
    \markcoordinate {#1} {lsl0}  {lpin 5}
    \markcoordinate {#1} {lsl1}  {lpin 4}
    \markcoordinate {#1} {lsl2}  {lpin 3}
    \markcoordinate {#1} {lsl3}  {lpin 2}
    %%%
    (#1.rpin 5) coordinate (#1-rsl0)
    (#1.rpin 4) coordinate (#1-rsl1)
    (#1.rpin 3) coordinate (#1-rsl2)
    (#1.rpin 2) coordinate (#1-rsl3)
    %%%
    \markcoordinate {#1} {in0}   {tpin 1}
    \markcoordinate {#1} {in1}   {tpin 2}
    \markcoordinate {#1} {out0}  {bpin 1}
    \markcoordinate {#1} {out1}  {bpin 2}
}

\ctikzsubcircuitactivate{subctkmuxFirstLayer}

\newcommand\muxFirstLayer[3] {
    \subctkmuxFirstLayer{#1} {#2}
    (#1.center) node[]{#3}
}

\ctikzsubcircuitdef{subctkmuxSecondLayer} {
    in00, in01, in02, in03, in04, in05, in06, in07,
    in08, in09, in10, in11, in12, in13, in14, in15,
    topfstmid, topsecmid,
    muxlast, out,
    bin00, bin01, bin02, bin03, bin04, bin05, bin06, bin07,
    bin08, bin09, bin10, bin11, bin12, bin13, bin14, bin15,
    bout%
} {
    node (#1) [muxSecondLayer]{}%
    \markcoordinate {#1} {in00}   {lpin 4}
    \markcoordinate {#1} {in01}   {lpin 3}
    \markcoordinate {#1} {in02}   {lpin 2}
    \markcoordinate {#1} {in03}   {lpin 1}
    %%%
    \markcoordinate {#1} {in04}   {tpin 1}
    \markcoordinate {#1} {in05}   {tpin 2}
    \markcoordinate {#1} {in06}   {tpin 3}
    \markcoordinate {#1} {in07}   {tpin 4}
    %%%
    \markcoordinate {#1} {in08}   {tpin 5}
    \markcoordinate {#1} {in09}   {tpin 6}
    \markcoordinate {#1} {in10}   {tpin 7}
    \markcoordinate {#1} {in11}   {tpin 8}
    %%%
    \markcoordinate {#1} {in12}   {rpin 1}
    \markcoordinate {#1} {in13}   {rpin 2}
    \markcoordinate {#1} {in14}   {rpin 3}
    \markcoordinate {#1} {in15}   {rpin 4}
    %%%
    \markcoordinate {#1} {out}     {bpin 2}
    %%%
    (#1.bpin 1) coordinate (#1-muxlast)
    %%%
    ($(#1-in05)!0.50!(#1-in06)$)
    coordinate (#1-topfstmid)
    ($(#1-in09)!0.50!(#1-in10)$)
    coordinate (#1-topsecmid)
    %%%
    ($(#1-in01)!0.50!(#1-in02)$)
    coordinate (#1-leftmid)
    ($(#1-in13)!0.50!(#1-in14)$)
    coordinate (#1-rightmid)
}

\ctikzsubcircuitactivate{subctkmuxSecondLayer}

\newcommand\muxSecondLayer[3] {
    \subctkmuxSecondLayer{#1} {#2}
    (#1.center) node[]{#3}
}

\ctikzsubcircuitdef{subctkmuxSixteen} {
    in00, in01, in02, in03, in04, in05, in06, in07,
    in08, in09, in10, in11, in12, in13, in14, in15,
    sl0, sl1, sl2, sl3,
    north, south, east, west,
    northeast, northwest, southeast, southwest,
    center,
    muxlast, out%
} {
    \muxFirstLayer {#1-fl07} {in0} {}
    (#1-fl07-in1)
    ++(2,0)
    \muxFirstLayer {#1-fl06} {in0} {}
    %%
    %% Drawing rest of the mux first layer
    %%
    (#1-fl06-in0) ++($(#1-fl06-in1)-(#1-fl07-in1)$)
    \muxFirstLayer {#1-fl05} {in0} {}
    (#1-fl05-in0) ++($(#1-fl06-in1)-(#1-fl07-in1)$)
    \muxFirstLayer {#1-fl04} {in0} {}
    (#1-fl04-in0) ++($(#1-fl06-in1)-(#1-fl07-in1)$) ++(2,0)
    \muxFirstLayer {#1-fl03} {in0} {}
    (#1-fl03-in0) ++($(#1-fl06-in1)-(#1-fl07-in1)$)
    \muxFirstLayer {#1-fl02} {in0} {}
    (#1-fl02-in0) ++($(#1-fl06-in1)-(#1-fl07-in1)$)
    \muxFirstLayer {#1-fl01} {in0} {}
    (#1-fl01-in0) ++($(#1-fl06-in1)-(#1-fl07-in1)$)
    \muxFirstLayer {#1-fl00} {in0} {}
    %%
    %% Marking second 4 fls to second layer lines
    %%
    ($(#1-fl05-out1)!0.50!(#1-fl04-out0)$)
    ++(0,-1.5)
    coordinate (#1-muxfl42midvrt)
    ++(0,-0.5)
    coordinate (#1-muxfl42midheight)
    %%
    ++(0,-5)
    \muxSecondLayer {#1-sl} {topfstmid} {}
    %%
    %% Drawing that layer
    %%
    (#1-fl05-out1)
    \varcornerdownright {(#1-muxfl42midheight-|#1-sl-in05)}
    -- (#1-sl-in05)
    %%
    (#1-fl04-out0)
    \varcornerdownleft {(#1-muxfl42midheight-|#1-sl-in06)}
    -- (#1-sl-in06)
    %%
    (#1-fl05-out0)
    \aroundcornertilldownright {($(#1-sl-in05)-(#1-sl-in04)$)}
    {(#1-muxfl42midheight-|#1-sl-in05)}
    {(#1-sl-in04)} {(0,1)}
    -- (#1-sl-in04)
    %%
    (#1-fl04-out1)
    \aroundcornertilldownleft {($(#1-sl-in06)-(#1-sl-in07)$)}
    {(#1-muxfl42midheight-|#1-sl-in06)}
    {(#1-sl-in07)} {(0,1)}
    -- (#1-sl-in07)
    %%
    %% Marking third 4 fls to second layer lines
    %%
    ($(#1-fl03-out1)!0.50!(#1-fl02-out0)$)
    ++(0,-1.5)
    coordinate (#1-muxfl34midvrt)
    ++(0,-0.5)
    coordinate (#1-muxfl34midheight)
    %%
    (#1-muxfl34midheight)
    ++($(#1-sl-in10)-(#1-sl-topsecmid)$)
    coordinate (#1-fl02out0)
    ++($(#1-sl-in11)-(#1-sl-in10)$)
    coordinate (#1-fl02out1)
    %%
    (#1-muxfl34midheight)
    ++($(#1-sl-in09)-(#1-sl-topsecmid)$)
    coordinate (#1-fl03out1)
    ++($(#1-sl-in08)-(#1-sl-in09)$)
    coordinate (#1-fl03out0)
    %%
    (#1-sl-in11)
    ++(0,1)
    coordinate (#1-muxfl3botfromslheight)
    %%
    %% Ofcourse, black magic
    \getmagnitude {#1} {($(#1-sl-in11)-(#1-sl-in08)$)}
    ++(0,\n{#1-magtmp})
    coordinate (#1-muxfl3topfromslheight)
    %%
    %%%
    %%
    (#1-fl02-out1)
    \aroundcornertilldownleft {($(#1-fl02out0)-(#1-fl02out1)$)}
    {(#1-fl02out0)}
    {(#1-fl02out1)} {(0,1)}
    %%
    \aroundcornertilldownleft {($(#1-fl03out0)-(#1-fl02out1)$)}
    {(#1-muxfl3topfromslheight-|#1-fl03out0) ++(-\cornerbase,0)}
    {(#1-muxfl3botfromslheight)} {(1,0)}
    \constcornerleft {(#1-sl-in11)}
    -- (#1-sl-in11)
    %%
    %%%
    %%
    (#1-fl02-out0)
    \varcornerdownleft {(#1-fl02out0)}
    %%
    \aroundcornerdownleft {($(#1-fl03out0)-(#1-fl02out0)$)}
    {(#1-muxfl3topfromslheight-|#1-fl03out0) ++(-\cornerbase,0)}
    %%
    \aroundcornertillleftdown {($(#1-fl02out0)-(#1-fl02out1)$)}
    {(#1-muxfl3botfromslheight-|#1-sl-in11) ++(0,-\cornerbase)}
    {(#1-sl-in10)} {(0,1)}
    %%
    -- (#1-sl-in10)
    %%
    %%%
    %%
    (#1-fl03-out1)
    \varcornerdownright {(#1-fl03out1)}
    %%
    \aroundcornerdownleft {($(#1-fl03out0)-(#1-fl03out1)$)}
    {(#1-muxfl3topfromslheight-|#1-fl03out0) ++(-\cornerbase,0)}
    %%
    \aroundcornertillleftdown {($(#1-fl03out1)-(#1-fl02out1)$)}
    {(#1-muxfl3botfromslheight-|#1-sl-in11) ++(0,-\cornerbase)}
    {(#1-sl-in09)} {(0,1)}
    %%
    -- (#1-sl-in09)
    %%
    %%%
    %%
    (#1-fl03-out0)
    \aroundcornertilldownright {($(#1-fl03out1)-(#1-fl03out0)$)}
    {(#1-fl03out1)}
    {(#1-fl03out0)} {(0,1)}
    %%
    \constcornerdown {(#1-fl03out0|-#1-muxfl3topfromslheight) ++(-1,0)}
    %%
    \aroundcornertillleftdown {($(#1-fl03out0)-(#1-fl02out1)$)}
    {(#1-muxfl3botfromslheight-|#1-sl-in11) ++(0,-\cornerbase)}
    {(#1-sl-in08)} {(0,1)}
    %%
    -- (#1-sl-in08)
    %%
    %%%
    %%
    %% Last fl
    %%
    ($(#1-fl01-out1)!0.50!(#1-fl00-out0)$)
    ++(0,-1.5)
    coordinate (#1-muxfl44midvrt)
    ++(0,-0.5)
    coordinate (#1-muxfl44midheight)
    %%
    (#1-muxfl44midheight)
    \getmagnitude{#1} {($(#1-sl-rightmid)-(#1-sl-in14)$)}
    ++(\n{#1-magtmp},0)
    coordinate (#1-fl00out0)
    \getmagnitude{#1} {($(#1-sl-in15)-(#1-sl-in14)$)}
    ++(\n{#1-magtmp},0)
    coordinate (#1-fl00out1)
    %%
    (#1-muxfl44midheight)
    \getmagnitude{#1} {($(#1-sl-rightmid)-(#1-sl-in13)$)}
    ++(-\n{#1-magtmp},0)
    coordinate (#1-fl01out1)
    \getmagnitude{#1} {($(#1-sl-in13)-(#1-sl-in12)$)}
    ++(-\n{#1-magtmp},0)
    coordinate (#1-fl01out0)
    %%
    %% Drawing last fl to sl lines
    %%
    (#1-fl01-out1)
    \varcornerdownright {(#1-fl01out1)}
    \aroundcornerdownleft {($(#1-fl01out0)-(#1-fl01out1)$)}
    {(#1-fl01out0|-#1-sl-in12) ++(-\cornerbase,0)}
    -- (#1-sl-in13)
    %%
    %%%
    %%
    (#1-fl00-out0)
    \varcornerdownleft {(#1-fl00out0)}
    \aroundcornerdownleft {($(#1-fl01out0)-(#1-fl00out0)$)}
    {(#1-fl01out0|-#1-sl-in12) ++(-\cornerbase,0)}
    -- (#1-sl-in14)
    %%
    %%%
    %%
    (#1-fl00-out1)
    \aroundcornertilldownleft {($(#1-fl00out0)-(#1-fl00out1)$)}
    {(#1-fl00out0)}
    {(#1-fl00out1)} {(0,1)}
    \aroundcornerdownleft {($(#1-fl01out0)-(#1-fl00out1)$)}
    {(#1-fl01out0|-#1-sl-in12) ++(-\cornerbase,0)}
    -- (#1-sl-in15)
    %%
    %%%
    %%
    (#1-fl01-out0)
    \aroundcornertilldownright {($(#1-fl01out1)-(#1-fl01out0)$)}
    {(#1-fl01out1)}
    {(#1-fl01out0)} {(0,1)}
    \constcornerdown {(#1-fl01out0|-#1-sl-in12)}
    -- (#1-sl-in12)
    %%
    %% Markings for First 4 fls
    %%
    ($(#1-fl07-out1)!0.50!(#1-fl06-out0)$)
    ++(0,-1.5)
    coordinate (#1-muxfl14midvrt)
    ++(0,-0.5)
    coordinate (#1-muxfl14midheight)
    %%
    (#1-muxfl14midheight)
    \getmagnitude{#1} {($(#1-sl-leftmid)-(#1-sl-in02)$)}
    ++(\n{#1-magtmp},0)
    coordinate (#1-fl06out0)
    \getmagnitude{#1} {($(#1-sl-in02)-(#1-sl-in03)$)}
    ++(\n{#1-magtmp},0)
    coordinate (#1-fl06out1)
    %%
    (#1-muxfl14midheight)
    \getmagnitude{#1} {($(#1-sl-leftmid)-(#1-sl-in01)$)}
    ++(-\n{#1-magtmp},0)
    coordinate (#1-fl07out1)
    \getmagnitude{#1} {($(#1-sl-in01)-(#1-sl-in00)$)}
    ++(-\n{#1-magtmp},0)
    coordinate (#1-fl07out0)
    %%
    %% Drawing the lines
    %%
    (#1-fl07-out1)
    \varcornerdownright {(#1-fl07out1)}
    \aroundcornerdownright {($(#1-fl06out1)-(#1-fl07out1)$)}
    {(#1-fl06out1|-#1-sl-in03) ++(\cornerbase,0)}
    -- (#1-sl-in01)
    %%
    %%%
    %%
    (#1-fl06-out0)
    \varcornerdownleft {(#1-fl06out0)}
    \aroundcornerdownright {($(#1-fl06out1)-(#1-fl06out0)$)}
    {(#1-fl06out1|-#1-sl-in03) ++(\cornerbase,0)}
    -- (#1-sl-in02)
    %%
    %%%
    %%
    (#1-fl06-out1)
    \aroundcornertilldownleft {($(#1-fl06out0)-(#1-fl06out1)$)}
    {(#1-fl06out0)}
    {(#1-fl06out1)} {(0,1)}
    %%
    \constcornerdown {(#1-sl-in03)}
    -- (#1-sl-in03)
    %%
    %%%
    %%
    (#1-fl07-out0)
    \aroundcornertilldownright {($(#1-fl07out1)-(#1-fl07out0)$)}
    {(#1-fl07out1)}
    {(#1-fl07out0)} {(0,1)}
    %%
    \aroundcornerdownright {($(#1-fl06out1)-(#1-fl07out0)$)}
    {(#1-fl06out1|-#1-sl-in03) ++(\cornerbase,0)}
    -- (#1-sl-in00)
    %%
    %% Drawing select lines
    %%
    %% First
    %%
    (#1-fl07-rsl0) -- (#1-fl06-lsl0)
    (#1-fl06-rsl0) -- (#1-fl05-lsl0)
    (#1-fl05-rsl0) -- (#1-fl04-lsl0)
    (#1-fl04-rsl0) -- (#1-fl03-lsl0)
    (#1-fl03-rsl0) -- (#1-fl02-lsl0)
    (#1-fl02-rsl0) -- (#1-fl01-lsl0)
    (#1-fl01-rsl0) -- (#1-fl00-lsl0)
    %%
    %% Second
    %%
    (#1-fl07-rsl1) -- (#1-fl06-lsl1)
    (#1-fl06-rsl1) -- (#1-fl05-lsl1)
    (#1-fl05-rsl1) -- (#1-fl04-lsl1)
    (#1-fl04-rsl1) -- (#1-fl03-lsl1)
    (#1-fl03-rsl1) -- (#1-fl02-lsl1)
    (#1-fl02-rsl1) -- (#1-fl01-lsl1)
    (#1-fl01-rsl1) -- (#1-fl00-lsl1)
    %%
    %% Third
    %%
    (#1-fl07-rsl2) -- (#1-fl06-lsl2)
    (#1-fl06-rsl2) -- (#1-fl05-lsl2)
    (#1-fl05-rsl2) -- (#1-fl04-lsl2)
    (#1-fl04-rsl2) -- (#1-fl03-lsl2)
    (#1-fl03-rsl2) -- (#1-fl02-lsl2)
    (#1-fl02-rsl2) -- (#1-fl01-lsl2)
    (#1-fl01-rsl2) -- (#1-fl00-lsl2)
    %%
    %% Fourth
    %%
    (#1-fl07-rsl3) -- (#1-fl06-lsl3)
    (#1-fl06-rsl3) -- (#1-fl05-lsl3)
    (#1-fl05-rsl3) -- (#1-fl04-lsl3)
    (#1-fl04-rsl3) -- (#1-fl03-lsl3)
    (#1-fl03-rsl3) -- (#1-fl02-lsl3)
    (#1-fl02-rsl3) -- (#1-fl01-lsl3)
    (#1-fl01-rsl3) -- (#1-fl00-lsl3)
    %%
    %% Drawing select terminals
    %%
    (#1-fl07-lsl0) -- ++(-3,0) coordinate (#1-sl0)
    (#1-fl07-lsl1) -- (#1-fl07-lsl1-|#1-sl0) coordinate (#1-sl1)
    (#1-fl07-lsl2) -- (#1-fl07-lsl2-|#1-sl0) coordinate (#1-sl2)
    (#1-fl07-lsl3) -- (#1-fl07-lsl3-|#1-sl0) coordinate (#1-sl3)
    %%
    %% Drawing inputs
    %%
    (#1-fl00-in0) -- ++(0,1) coordinate (#1-in01)
    (#1-fl00-in1) -- (#1-in01-|#1-fl00-in1) coordinate (#1-in00)
    %%
    (#1-fl00-in0) -- (#1-in00-|#1-fl00-in0) coordinate (#1-in01)
    (#1-fl01-in0) -- (#1-in00-|#1-fl01-in0) coordinate (#1-in03)
    (#1-fl02-in0) -- (#1-in00-|#1-fl02-in0) coordinate (#1-in05)
    (#1-fl03-in0) -- (#1-in00-|#1-fl03-in0) coordinate (#1-in07)
    (#1-fl04-in0) -- (#1-in00-|#1-fl04-in0) coordinate (#1-in09)
    (#1-fl05-in0) -- (#1-in00-|#1-fl05-in0) coordinate (#1-in11)
    (#1-fl06-in0) -- (#1-in00-|#1-fl06-in0) coordinate (#1-in13)
    (#1-fl07-in0) -- (#1-in00-|#1-fl07-in0) coordinate (#1-in15)
    %%
    (#1-fl00-in1) -- (#1-in00-|#1-fl00-in1) coordinate (#1-in00)
    (#1-fl01-in1) -- (#1-in00-|#1-fl01-in1) coordinate (#1-in02)
    (#1-fl02-in1) -- (#1-in00-|#1-fl02-in1) coordinate (#1-in04)
    (#1-fl03-in1) -- (#1-in00-|#1-fl03-in1) coordinate (#1-in06)
    (#1-fl04-in1) -- (#1-in00-|#1-fl04-in1) coordinate (#1-in08)
    (#1-fl05-in1) -- (#1-in00-|#1-fl05-in1) coordinate (#1-in10)
    (#1-fl06-in1) -- (#1-in00-|#1-fl06-in1) coordinate (#1-in12)
    (#1-fl07-in1) -- (#1-in00-|#1-fl07-in1) coordinate (#1-in14)
    %%
    %% Drawing outputs
    %%
    (#1-sl-out)
    -- ++(0,-1)
    coordinate (#1-out)
    %%
    (#1-sl-muxlast)
    coordinate (#1-muxlast)
    %%
    %% Drawing Muxlast is not always needed therefore
    %% create a newcommand to do it
    \markgeocoordinate {#1}
    {(#1-in00)} {(#1-out)}
    {(#1-sl0)} {(#1-fl00-rsl0)}
    %%
}

\ctikzsubcircuitactivate{subctkmuxSixteen}

\newcommand\muxSixteen[2] {
    \subctkmuxSixteen{#1} {#2}
}

\newcommand\labelmuxSixteenFl[9] {
    (#1-fl07.center) node [] {#2}
    (#1-fl06.center) node [] {#3}
    (#1-fl05.center) node [] {#4}
    (#1-fl04.center) node [] {#5}
    (#1-fl03.center) node [] {#6}
    (#1-fl02.center) node [] {#7}
    (#1-fl01.center) node [] {#8}
    (#1-fl00.center) node [] {#9}
}

\newcommand\labelmuxSixteenSl[2] {
    (#1-sl.center) node [] {#2}
}

    %% um, why did I add this...
    %% % \tikzset{clkGen/.style = {muxdemux, muxdemux def = {
%     NL = 3, NR = 4, NB = 0, NT = 0,
%     Lh = 3, Rh = 3,
%     w = 5 },
%     draw only left pins = {1} }
% }

\tikzset{
    muxtest/.style = {
        muxdemux, muxdemux def={NL=16, NR=3, NT=5, NB=3, w=2,
            inset w=0.5, Lh=4, inset Lh=2.0, inset Rh=1.0,
            square pins=1},
        draw only right pins={1,3},
        draw only top pins={1-3},
        draw only bottom pins={3}
    }
}

\tikzset{mux16/.style = {muxdemux, muxdemux def = {
    NL = 21, NR = 3, NB = 6, NT = 0,
    Lh = 42, Rh = 27,
    inset Lh = 0, inset Rh = 0,
    inset w = 0,
    %% inset Lh = 5, inset Rh = 2.75,
    %% inset w = 1.25,
    w = 9,
    square pins = 1
    },
    draw only left pins={2-5,7-10,12-15,17-20},
    draw only right pins={2},
    draw only bottom pins={2-5}
    }
}

    %% \ctikzsubcircuitdef{subctkmuxFirstLayer} {
    in0, in1, out0, out1,
    lsl0, lsl1, lsl2, lsl3,
    rsl0, rsl1, rsl2, rsl3,
    blsl0, blsl1, blsl2, blsl3,
    bin0, bin1, bout0, bout1%
} {
    node (#1) [muxFirstLayer]{}%
    \markcoordinate {#1} {lsl0}  {lpin 5}
    \markcoordinate {#1} {lsl1}  {lpin 4}
    \markcoordinate {#1} {lsl2}  {lpin 3}
    \markcoordinate {#1} {lsl3}  {lpin 2}
    %%%
    (#1.rpin 5) coordinate (#1-rsl0)
    (#1.rpin 4) coordinate (#1-rsl1)
    (#1.rpin 3) coordinate (#1-rsl2)
    (#1.rpin 2) coordinate (#1-rsl3)
    %%%
    \markcoordinate {#1} {in0}   {tpin 1}
    \markcoordinate {#1} {in1}   {tpin 2}
    \markcoordinate {#1} {out0}  {bpin 1}
    \markcoordinate {#1} {out1}  {bpin 2}
}

\ctikzsubcircuitactivate{subctkmuxFirstLayer}

\newcommand\muxFirstLayer[3] {
    \subctkmuxFirstLayer{#1} {#2}
    (#1.center) node[]{#3}
}

\ctikzsubcircuitdef{subctkmuxSecondLayer} {
    in00, in01, in02, in03, in04, in05, in06, in07,
    in08, in09, in10, in11, in12, in13, in14, in15,
    topfstmid, topsecmid,
    muxlast, out,
    bin00, bin01, bin02, bin03, bin04, bin05, bin06, bin07,
    bin08, bin09, bin10, bin11, bin12, bin13, bin14, bin15,
    bout%
} {
    node (#1) [muxSecondLayer]{}%
    \markcoordinate {#1} {in00}   {lpin 4}
    \markcoordinate {#1} {in01}   {lpin 3}
    \markcoordinate {#1} {in02}   {lpin 2}
    \markcoordinate {#1} {in03}   {lpin 1}
    %%%
    \markcoordinate {#1} {in04}   {tpin 1}
    \markcoordinate {#1} {in05}   {tpin 2}
    \markcoordinate {#1} {in06}   {tpin 3}
    \markcoordinate {#1} {in07}   {tpin 4}
    %%%
    \markcoordinate {#1} {in08}   {tpin 5}
    \markcoordinate {#1} {in09}   {tpin 6}
    \markcoordinate {#1} {in10}   {tpin 7}
    \markcoordinate {#1} {in11}   {tpin 8}
    %%%
    \markcoordinate {#1} {in12}   {rpin 1}
    \markcoordinate {#1} {in13}   {rpin 2}
    \markcoordinate {#1} {in14}   {rpin 3}
    \markcoordinate {#1} {in15}   {rpin 4}
    %%%
    \markcoordinate {#1} {out}     {bpin 2}
    %%%
    (#1.bpin 1) coordinate (#1-muxlast)
    %%%
    ($(#1-in05)!0.50!(#1-in06)$)
    coordinate (#1-topfstmid)
    ($(#1-in09)!0.50!(#1-in10)$)
    coordinate (#1-topsecmid)
    %%%
    ($(#1-in01)!0.50!(#1-in02)$)
    coordinate (#1-leftmid)
    ($(#1-in13)!0.50!(#1-in14)$)
    coordinate (#1-rightmid)
}

\ctikzsubcircuitactivate{subctkmuxSecondLayer}

\newcommand\muxSecondLayer[3] {
    \subctkmuxSecondLayer{#1} {#2}
    (#1.center) node[]{#3}
}

\ctikzsubcircuitdef{subctkmuxSixteen} {
    in00, in01, in02, in03, in04, in05, in06, in07,
    in08, in09, in10, in11, in12, in13, in14, in15,
    sl0, sl1, sl2, sl3,
    north, south, east, west,
    northeast, northwest, southeast, southwest,
    center,
    muxlast, out%
} {
    \muxFirstLayer {#1-fl07} {in0} {}
    (#1-fl07-in1)
    ++(2,0)
    \muxFirstLayer {#1-fl06} {in0} {}
    %%
    %% Drawing rest of the mux first layer
    %%
    (#1-fl06-in0) ++($(#1-fl06-in1)-(#1-fl07-in1)$)
    \muxFirstLayer {#1-fl05} {in0} {}
    (#1-fl05-in0) ++($(#1-fl06-in1)-(#1-fl07-in1)$)
    \muxFirstLayer {#1-fl04} {in0} {}
    (#1-fl04-in0) ++($(#1-fl06-in1)-(#1-fl07-in1)$) ++(2,0)
    \muxFirstLayer {#1-fl03} {in0} {}
    (#1-fl03-in0) ++($(#1-fl06-in1)-(#1-fl07-in1)$)
    \muxFirstLayer {#1-fl02} {in0} {}
    (#1-fl02-in0) ++($(#1-fl06-in1)-(#1-fl07-in1)$)
    \muxFirstLayer {#1-fl01} {in0} {}
    (#1-fl01-in0) ++($(#1-fl06-in1)-(#1-fl07-in1)$)
    \muxFirstLayer {#1-fl00} {in0} {}
    %%
    %% Marking second 4 fls to second layer lines
    %%
    ($(#1-fl05-out1)!0.50!(#1-fl04-out0)$)
    ++(0,-1.5)
    coordinate (#1-muxfl42midvrt)
    ++(0,-0.5)
    coordinate (#1-muxfl42midheight)
    %%
    ++(0,-5)
    \muxSecondLayer {#1-sl} {topfstmid} {}
    %%
    %% Drawing that layer
    %%
    (#1-fl05-out1)
    \varcornerdownright {(#1-muxfl42midheight-|#1-sl-in05)}
    -- (#1-sl-in05)
    %%
    (#1-fl04-out0)
    \varcornerdownleft {(#1-muxfl42midheight-|#1-sl-in06)}
    -- (#1-sl-in06)
    %%
    (#1-fl05-out0)
    \aroundcornertilldownright {($(#1-sl-in05)-(#1-sl-in04)$)}
    {(#1-muxfl42midheight-|#1-sl-in05)}
    {(#1-sl-in04)} {(0,1)}
    -- (#1-sl-in04)
    %%
    (#1-fl04-out1)
    \aroundcornertilldownleft {($(#1-sl-in06)-(#1-sl-in07)$)}
    {(#1-muxfl42midheight-|#1-sl-in06)}
    {(#1-sl-in07)} {(0,1)}
    -- (#1-sl-in07)
    %%
    %% Marking third 4 fls to second layer lines
    %%
    ($(#1-fl03-out1)!0.50!(#1-fl02-out0)$)
    ++(0,-1.5)
    coordinate (#1-muxfl34midvrt)
    ++(0,-0.5)
    coordinate (#1-muxfl34midheight)
    %%
    (#1-muxfl34midheight)
    ++($(#1-sl-in10)-(#1-sl-topsecmid)$)
    coordinate (#1-fl02out0)
    ++($(#1-sl-in11)-(#1-sl-in10)$)
    coordinate (#1-fl02out1)
    %%
    (#1-muxfl34midheight)
    ++($(#1-sl-in09)-(#1-sl-topsecmid)$)
    coordinate (#1-fl03out1)
    ++($(#1-sl-in08)-(#1-sl-in09)$)
    coordinate (#1-fl03out0)
    %%
    (#1-sl-in11)
    ++(0,1)
    coordinate (#1-muxfl3botfromslheight)
    %%
    %% Ofcourse, black magic
    \getmagnitude {#1} {($(#1-sl-in11)-(#1-sl-in08)$)}
    ++(0,\n{#1-magtmp})
    coordinate (#1-muxfl3topfromslheight)
    %%
    %%%
    %%
    (#1-fl02-out1)
    \aroundcornertilldownleft {($(#1-fl02out0)-(#1-fl02out1)$)}
    {(#1-fl02out0)}
    {(#1-fl02out1)} {(0,1)}
    %%
    \aroundcornertilldownleft {($(#1-fl03out0)-(#1-fl02out1)$)}
    {(#1-muxfl3topfromslheight-|#1-fl03out0) ++(-\cornerbase,0)}
    {(#1-muxfl3botfromslheight)} {(1,0)}
    \constcornerleft {(#1-sl-in11)}
    -- (#1-sl-in11)
    %%
    %%%
    %%
    (#1-fl02-out0)
    \varcornerdownleft {(#1-fl02out0)}
    %%
    \aroundcornerdownleft {($(#1-fl03out0)-(#1-fl02out0)$)}
    {(#1-muxfl3topfromslheight-|#1-fl03out0) ++(-\cornerbase,0)}
    %%
    \aroundcornertillleftdown {($(#1-fl02out0)-(#1-fl02out1)$)}
    {(#1-muxfl3botfromslheight-|#1-sl-in11) ++(0,-\cornerbase)}
    {(#1-sl-in10)} {(0,1)}
    %%
    -- (#1-sl-in10)
    %%
    %%%
    %%
    (#1-fl03-out1)
    \varcornerdownright {(#1-fl03out1)}
    %%
    \aroundcornerdownleft {($(#1-fl03out0)-(#1-fl03out1)$)}
    {(#1-muxfl3topfromslheight-|#1-fl03out0) ++(-\cornerbase,0)}
    %%
    \aroundcornertillleftdown {($(#1-fl03out1)-(#1-fl02out1)$)}
    {(#1-muxfl3botfromslheight-|#1-sl-in11) ++(0,-\cornerbase)}
    {(#1-sl-in09)} {(0,1)}
    %%
    -- (#1-sl-in09)
    %%
    %%%
    %%
    (#1-fl03-out0)
    \aroundcornertilldownright {($(#1-fl03out1)-(#1-fl03out0)$)}
    {(#1-fl03out1)}
    {(#1-fl03out0)} {(0,1)}
    %%
    \constcornerdown {(#1-fl03out0|-#1-muxfl3topfromslheight) ++(-1,0)}
    %%
    \aroundcornertillleftdown {($(#1-fl03out0)-(#1-fl02out1)$)}
    {(#1-muxfl3botfromslheight-|#1-sl-in11) ++(0,-\cornerbase)}
    {(#1-sl-in08)} {(0,1)}
    %%
    -- (#1-sl-in08)
    %%
    %%%
    %%
    %% Last fl
    %%
    ($(#1-fl01-out1)!0.50!(#1-fl00-out0)$)
    ++(0,-1.5)
    coordinate (#1-muxfl44midvrt)
    ++(0,-0.5)
    coordinate (#1-muxfl44midheight)
    %%
    (#1-muxfl44midheight)
    \getmagnitude{#1} {($(#1-sl-rightmid)-(#1-sl-in14)$)}
    ++(\n{#1-magtmp},0)
    coordinate (#1-fl00out0)
    \getmagnitude{#1} {($(#1-sl-in15)-(#1-sl-in14)$)}
    ++(\n{#1-magtmp},0)
    coordinate (#1-fl00out1)
    %%
    (#1-muxfl44midheight)
    \getmagnitude{#1} {($(#1-sl-rightmid)-(#1-sl-in13)$)}
    ++(-\n{#1-magtmp},0)
    coordinate (#1-fl01out1)
    \getmagnitude{#1} {($(#1-sl-in13)-(#1-sl-in12)$)}
    ++(-\n{#1-magtmp},0)
    coordinate (#1-fl01out0)
    %%
    %% Drawing last fl to sl lines
    %%
    (#1-fl01-out1)
    \varcornerdownright {(#1-fl01out1)}
    \aroundcornerdownleft {($(#1-fl01out0)-(#1-fl01out1)$)}
    {(#1-fl01out0|-#1-sl-in12) ++(-\cornerbase,0)}
    -- (#1-sl-in13)
    %%
    %%%
    %%
    (#1-fl00-out0)
    \varcornerdownleft {(#1-fl00out0)}
    \aroundcornerdownleft {($(#1-fl01out0)-(#1-fl00out0)$)}
    {(#1-fl01out0|-#1-sl-in12) ++(-\cornerbase,0)}
    -- (#1-sl-in14)
    %%
    %%%
    %%
    (#1-fl00-out1)
    \aroundcornertilldownleft {($(#1-fl00out0)-(#1-fl00out1)$)}
    {(#1-fl00out0)}
    {(#1-fl00out1)} {(0,1)}
    \aroundcornerdownleft {($(#1-fl01out0)-(#1-fl00out1)$)}
    {(#1-fl01out0|-#1-sl-in12) ++(-\cornerbase,0)}
    -- (#1-sl-in15)
    %%
    %%%
    %%
    (#1-fl01-out0)
    \aroundcornertilldownright {($(#1-fl01out1)-(#1-fl01out0)$)}
    {(#1-fl01out1)}
    {(#1-fl01out0)} {(0,1)}
    \constcornerdown {(#1-fl01out0|-#1-sl-in12)}
    -- (#1-sl-in12)
    %%
    %% Markings for First 4 fls
    %%
    ($(#1-fl07-out1)!0.50!(#1-fl06-out0)$)
    ++(0,-1.5)
    coordinate (#1-muxfl14midvrt)
    ++(0,-0.5)
    coordinate (#1-muxfl14midheight)
    %%
    (#1-muxfl14midheight)
    \getmagnitude{#1} {($(#1-sl-leftmid)-(#1-sl-in02)$)}
    ++(\n{#1-magtmp},0)
    coordinate (#1-fl06out0)
    \getmagnitude{#1} {($(#1-sl-in02)-(#1-sl-in03)$)}
    ++(\n{#1-magtmp},0)
    coordinate (#1-fl06out1)
    %%
    (#1-muxfl14midheight)
    \getmagnitude{#1} {($(#1-sl-leftmid)-(#1-sl-in01)$)}
    ++(-\n{#1-magtmp},0)
    coordinate (#1-fl07out1)
    \getmagnitude{#1} {($(#1-sl-in01)-(#1-sl-in00)$)}
    ++(-\n{#1-magtmp},0)
    coordinate (#1-fl07out0)
    %%
    %% Drawing the lines
    %%
    (#1-fl07-out1)
    \varcornerdownright {(#1-fl07out1)}
    \aroundcornerdownright {($(#1-fl06out1)-(#1-fl07out1)$)}
    {(#1-fl06out1|-#1-sl-in03) ++(\cornerbase,0)}
    -- (#1-sl-in01)
    %%
    %%%
    %%
    (#1-fl06-out0)
    \varcornerdownleft {(#1-fl06out0)}
    \aroundcornerdownright {($(#1-fl06out1)-(#1-fl06out0)$)}
    {(#1-fl06out1|-#1-sl-in03) ++(\cornerbase,0)}
    -- (#1-sl-in02)
    %%
    %%%
    %%
    (#1-fl06-out1)
    \aroundcornertilldownleft {($(#1-fl06out0)-(#1-fl06out1)$)}
    {(#1-fl06out0)}
    {(#1-fl06out1)} {(0,1)}
    %%
    \constcornerdown {(#1-sl-in03)}
    -- (#1-sl-in03)
    %%
    %%%
    %%
    (#1-fl07-out0)
    \aroundcornertilldownright {($(#1-fl07out1)-(#1-fl07out0)$)}
    {(#1-fl07out1)}
    {(#1-fl07out0)} {(0,1)}
    %%
    \aroundcornerdownright {($(#1-fl06out1)-(#1-fl07out0)$)}
    {(#1-fl06out1|-#1-sl-in03) ++(\cornerbase,0)}
    -- (#1-sl-in00)
    %%
    %% Drawing select lines
    %%
    %% First
    %%
    (#1-fl07-rsl0) -- (#1-fl06-lsl0)
    (#1-fl06-rsl0) -- (#1-fl05-lsl0)
    (#1-fl05-rsl0) -- (#1-fl04-lsl0)
    (#1-fl04-rsl0) -- (#1-fl03-lsl0)
    (#1-fl03-rsl0) -- (#1-fl02-lsl0)
    (#1-fl02-rsl0) -- (#1-fl01-lsl0)
    (#1-fl01-rsl0) -- (#1-fl00-lsl0)
    %%
    %% Second
    %%
    (#1-fl07-rsl1) -- (#1-fl06-lsl1)
    (#1-fl06-rsl1) -- (#1-fl05-lsl1)
    (#1-fl05-rsl1) -- (#1-fl04-lsl1)
    (#1-fl04-rsl1) -- (#1-fl03-lsl1)
    (#1-fl03-rsl1) -- (#1-fl02-lsl1)
    (#1-fl02-rsl1) -- (#1-fl01-lsl1)
    (#1-fl01-rsl1) -- (#1-fl00-lsl1)
    %%
    %% Third
    %%
    (#1-fl07-rsl2) -- (#1-fl06-lsl2)
    (#1-fl06-rsl2) -- (#1-fl05-lsl2)
    (#1-fl05-rsl2) -- (#1-fl04-lsl2)
    (#1-fl04-rsl2) -- (#1-fl03-lsl2)
    (#1-fl03-rsl2) -- (#1-fl02-lsl2)
    (#1-fl02-rsl2) -- (#1-fl01-lsl2)
    (#1-fl01-rsl2) -- (#1-fl00-lsl2)
    %%
    %% Fourth
    %%
    (#1-fl07-rsl3) -- (#1-fl06-lsl3)
    (#1-fl06-rsl3) -- (#1-fl05-lsl3)
    (#1-fl05-rsl3) -- (#1-fl04-lsl3)
    (#1-fl04-rsl3) -- (#1-fl03-lsl3)
    (#1-fl03-rsl3) -- (#1-fl02-lsl3)
    (#1-fl02-rsl3) -- (#1-fl01-lsl3)
    (#1-fl01-rsl3) -- (#1-fl00-lsl3)
    %%
    %% Drawing select terminals
    %%
    (#1-fl07-lsl0) -- ++(-3,0) coordinate (#1-sl0)
    (#1-fl07-lsl1) -- (#1-fl07-lsl1-|#1-sl0) coordinate (#1-sl1)
    (#1-fl07-lsl2) -- (#1-fl07-lsl2-|#1-sl0) coordinate (#1-sl2)
    (#1-fl07-lsl3) -- (#1-fl07-lsl3-|#1-sl0) coordinate (#1-sl3)
    %%
    %% Drawing inputs
    %%
    (#1-fl00-in0) -- ++(0,1) coordinate (#1-in01)
    (#1-fl00-in1) -- (#1-in01-|#1-fl00-in1) coordinate (#1-in00)
    %%
    (#1-fl00-in0) -- (#1-in00-|#1-fl00-in0) coordinate (#1-in01)
    (#1-fl01-in0) -- (#1-in00-|#1-fl01-in0) coordinate (#1-in03)
    (#1-fl02-in0) -- (#1-in00-|#1-fl02-in0) coordinate (#1-in05)
    (#1-fl03-in0) -- (#1-in00-|#1-fl03-in0) coordinate (#1-in07)
    (#1-fl04-in0) -- (#1-in00-|#1-fl04-in0) coordinate (#1-in09)
    (#1-fl05-in0) -- (#1-in00-|#1-fl05-in0) coordinate (#1-in11)
    (#1-fl06-in0) -- (#1-in00-|#1-fl06-in0) coordinate (#1-in13)
    (#1-fl07-in0) -- (#1-in00-|#1-fl07-in0) coordinate (#1-in15)
    %%
    (#1-fl00-in1) -- (#1-in00-|#1-fl00-in1) coordinate (#1-in00)
    (#1-fl01-in1) -- (#1-in00-|#1-fl01-in1) coordinate (#1-in02)
    (#1-fl02-in1) -- (#1-in00-|#1-fl02-in1) coordinate (#1-in04)
    (#1-fl03-in1) -- (#1-in00-|#1-fl03-in1) coordinate (#1-in06)
    (#1-fl04-in1) -- (#1-in00-|#1-fl04-in1) coordinate (#1-in08)
    (#1-fl05-in1) -- (#1-in00-|#1-fl05-in1) coordinate (#1-in10)
    (#1-fl06-in1) -- (#1-in00-|#1-fl06-in1) coordinate (#1-in12)
    (#1-fl07-in1) -- (#1-in00-|#1-fl07-in1) coordinate (#1-in14)
    %%
    %% Drawing outputs
    %%
    (#1-sl-out)
    -- ++(0,-1)
    coordinate (#1-out)
    %%
    (#1-sl-muxlast)
    coordinate (#1-muxlast)
    %%
    %% Drawing Muxlast is not always needed therefore
    %% create a newcommand to do it
    \markgeocoordinate {#1}
    {(#1-in00)} {(#1-out)}
    {(#1-sl0)} {(#1-fl00-rsl0)}
    %%
}

\ctikzsubcircuitactivate{subctkmuxSixteen}

\newcommand\muxSixteen[2] {
    \subctkmuxSixteen{#1} {#2}
}

\newcommand\labelmuxSixteenFl[9] {
    (#1-fl07.center) node [] {#2}
    (#1-fl06.center) node [] {#3}
    (#1-fl05.center) node [] {#4}
    (#1-fl04.center) node [] {#5}
    (#1-fl03.center) node [] {#6}
    (#1-fl02.center) node [] {#7}
    (#1-fl01.center) node [] {#8}
    (#1-fl00.center) node [] {#9}
}

\newcommand\labelmuxSixteenSl[2] {
    (#1-sl.center) node [] {#2}
}

    %% \tikzexternalize[prefix=figcache/]
    %% \usepackage{catppuccintheme}
    %% \usemintedstyle{catppuccin-mocha}
    %% \ctikzset{open poles fill = black}
    %% \setminted[verilog]{frame = lines, framesep = 2mm}
} {
}

\begin{document}

%%%% Some definitions

\def\truthTableWidth{0.4\textwidth}
\renewcommand\tabularxcolumn[1]{m{#1}}%
\newcolumntype{G}{>{\centering\arraybackslash}X}
\newcolumntype{T}{>{\centering\arraybackslash}m{\truthTableWidth}}

%%%%

\chapter{Basic Circuits}

\section {FlipFlops} \label{sec:flipflops}

\subsection {SR Latch} \label{sec:srlatch}

\begin{figure} [!ht]

    \renewcommand\tabularxcolumn[1]{m{#1}}%
    %% \newcolumntype{G}{>{\centering\arraybackslash}X}
    %% \newcolumntype{T}{>{\centering\arraybackslash}m{0.3\textwidth}}

    \begin{tabularx} {\textwidth} {*{1}G*{1}G}
        \centering
        \vfill
        \resizebox {0.3\textwidth} {!} {
            \includegraphics {tikzpics/epicglvlsrlatch.pdf}
            %% where here: epicglvlsrlatch
        }
        \vfill
        &
        %% \begin{tabularx} {>{\centering}m{0.5\linewidth}m{0.5\linewidth}}
        \vfill
        {\begin{tabularx} {\truthTableWidth} {*{3}{>{\centering\arraybackslash}X}}
            \toprule
            \textsc{\textbf{S}} & \textsc{\textbf{R}} & \textsc{\textbf{Q\textsubscript{$t+1$}}} \\
            \midrule
            0   &  0  &   Q\textsubscript{t} \\
            0   &  1  &       0 \\
            1   &  0  &       1 \\
            1   &  1  &    $Invalid$\\
            \bottomrule
        \end{tabularx}}
        \vfill
        \\
        \captionof{figure}{SR Latch}
        \label{fig:glvlsrlatch}
        &
        \captionof{table}{Truth Table of SR Latch}
        \label{tbl:ttsrlatch} \\
        \end{tabularx}

\end{figure}

Table: \ref{tbl:ttsrlatch} and Figure: \ref{fig:glvlsrlatch}

\subsubsection{Verilog Implementation} \label{hdl:srlatch}

\begin{minted}[mathescape] {verilog}
    module sr_latch(input logic set, reset,
        output logic q, qbar);

        nor #3 (q, qbar, reset);
        nor #3 (qbar, q, set);

    endmodule
\end{minted}

\subsection {SR FlipFlop} \label{sec:srflipflop}

\begin{figure} [!h]
    \begin{tabularx} {\textwidth} {*{1}G*{1}T}
        \centering
        \vfill
        \resizebox{0.5\textwidth}{!} {
            \includegraphics {tikzpics/epicglvlsrflipflop.pdf}
            %% where here: epicglvlsrflipflop
        }
        \vfill
        &
        \vfill
        {\begin{tabularx} {\truthTableWidth} {*{4}{>{\centering\arraybackslash}X}}
            \toprule
            \textsc{\textbf{CLK}} & \textsc{\textbf{S}} & \textsc{\textbf{R}} & \textsc{\textbf{Q\textsubscript{$t+1$}}} \\
            \midrule
            0   &   0   &  0  &    $Invalid$ \\
            0   &   0   &  1  &       0 \\
            0   &   1   &  0  &       1 \\
            0   &   1   &  1  &    Q\textsubscript{t} \\
            1   &   X   &  X  &    Q\textsubscript{t} \\
            \bottomrule
        \end{tabularx}}
        \vfill
        \\
        \caption{SR FlipFlop}
        \label{fig:glvlsrflipflop}
        &
        \captionof{table}{Truth Table of SR FlipFlop}
        \label{tbl:ttsrflipflop} \\
    \end{tabularx}
\end{figure}

\subsubsection{Verilog Implementation} \label{hdl:srflipflop}

\begin{minted}[mathescape] {verilog}
    module sr_flipflop(input logic set, reset, clk,
        output logic q, qbar);

        /*
         * With active low clk
         * */

        wire srlset, srlreset;

        /* Refer, (section $\ref{hdl:srlatch}$) */
        sr_latch srl (
            .set(srlset),
            .reset(srlreset),
            .q(q),
            .qbar(qbar)
        );

        nor #3 (srlreset, set, clk);
        nor #3 (srlset, reset, clk);

    endmodule
\end{minted}

\subsection {D FlipFlop} \label{sec:dflipflop}

\begin{figure} [!hb]
    \def\truthTableWidth{0.35\textwidth}
    \begin{tabularx} {\textwidth} {GT}
        \centering
        \vfill
        \resizebox{0.6\textwidth}{!} {
            \includegraphics {tikzpics/epicglvldflipflop.pdf}
            %% where here: epicglvldflipflop
        }
        \vfill
        &
        \vfill
        {\begin{tabularx} {0.75\linewidth} {*{3}{>{\centering\arraybackslash}X}}
            \toprule
            \textsc{\textbf{CLK}} & \textsc{\textbf{D}} & \textsc{\textbf{Q\textsubscript{t+1}}} \\
            \midrule
            0   &   0   &   0   \\
            0   &   1   &   1   \\
            1   &   X   &   Q\textsubscript{t}   \\
            \bottomrule
        \end{tabularx}}
        \vfill
        \\
        \captionof{figure}{D FlipFlop}
        \label{fig:glvldflipflop}
        &
        \captionof{table}{Truth Table of D FlipFlop}
        \label{tbl:dflipflop}
    \end{tabularx}
\end{figure}

\subsubsection{Verilog Implementation} \label{hdl:dflipflop}

\begin{minted}[mathescape] {verilog}
    module d_flipflop(input logic data, clk,
        output logic q, qbar);

        /*
         * With active low clk
         * */

         wire notdata;

        /* Refer, (section $\ref{hdl:srflipflop}$) */
        sr_flipflop srff (
            .set(data),
            .reset(notdata),
            .clk(clk),
            .q(q),
            .qbar(qbar)
        );

        nor #3 (notdata, data, data);

    endmodule
\end{minted}

\subsection {D FlipFlop with Synchronous Reset} \label{sec:dflipflopsr}

\begin{figure} [!ht]
    \def\truthTableWidth{0.35\textwidth}
    \begin{tabularx} {\textwidth} {GT}
        \centering
        \vfill
        \resizebox{0.6\textwidth}{!} {
            \includegraphics {tikzpics/epicglvldflipflopsr.pdf}
            %% where here: epicglvldflipflopsr
        }
        \vfill
        &
        \vfill
        {\begin{tabularx} {0.75\linewidth} {*{4}{>{\centering\arraybackslash}X}}
            \toprule
            \textsc{\textbf{CLK}} & \textsc{\textbf{RST}} & \textsc{\textbf{D}} & \textsc{\textbf{Q\textsubscript{t+1}}} \\
            \midrule
            0   &   0   &   0   &   1   \\
            0   &   0   &   1   &   0   \\
            0   &   1   &   X   &   0   \\
            1   &   X   &   X   &   Q\textsubscript{t}   \\
            \bottomrule
        \end{tabularx}}
        \vfill
        \\
        \caption{D FlipFlop with Synchronous Reset}
        \label{fig:glvldflipflopsr}
        &
        \captionof{table}{Truth Table of D FlipFlop with Synchronous Reset}
        \label{tbl:dflipflopsr}
    \end{tabularx}
\end{figure}

\subsubsection{Verilog Implementation} \label{hdl:dflipflopsr}

\begin{minted}[mathescape] {verilog}
    module d_flipflop_sr(input logic data, rst, clk,
        output logic q, qbar);

        /*
         * With active low clk and active low data
         * */

        wire acdata;

        /* Refer, (section $\ref{hdl:dflipflop}$) */
        d_flipflop dl (
            .data(acdata),
            .clk(clk),
            .q(q),
            .qbar(qbar)
        );

        nor #3 (acdata, data, rst);

    endmodule
\end{minted}

\subsection {MS FlipFlop} \label{sec:msflipflop}

\begin{figure} [!hb]
    \centering
    \resizebox{\textwidth}{!} {
        \includegraphics {tikzpics/epicglvlmsflipflop.pdf}
        %% where here: epicglvlmsflipflop
    }
    \caption{MS FlipFlop}
    \label{fig:glvlmsflipflop}
\end{figure}

\subsubsection{Verilog Implementation} \label{hdl:msflipflop}

\begin{minted}[mathescape] {verilog}
    module ms_flipflop(input logic data, rst, reclk, feclk,
        output logic q, qbar);

        /*
         * With synchronous reset, active low data,
         * active low reclk, and feclk
         * */

        wire sdata, notsdata;

        /* Refer, (section $\ref{hdl:dflipflopsr}$) */
        d_flipflop_sr master (
            .data(data),
            .clk(reclk),
            .q(sdata),
            .qbar(notsdata),
            .rst(rst)
        );

        /* Refer, (section $\ref{hdl:srflipflop}$) */
        sr_flipflop slave (
            .set(sdata),
            .reset(notsdata),
            .clk(feclk),
            .q(q),
            .qbar(qbar)
        );

    endmodule
\end{minted}

%% \end{document}

\section{Multiplexer}

\subsection {Five Input NOR Gate from Two Input NOR Gate}

\begin{minted} {verilog}
    module nor5input(input logic in0, in1, in2, in3, in4,
        output logic out, f4high);

        wire nor01, nor23;
        wire or01, or23;
        wire nor0123;
        wire or0123;

        nor #3 (nor01, in0, in1);
        nor #3 (or01, nor01, nor01);

        nor #3 (nor23, in2, in3);
        nor #3 (or23, nor23, nor23);

        nor #3 (nor0123, or01, or23);
        nor #3 (or0123, nor0123, nor0123);

        nor #3 (out, or0123, in4);

        assign f4high = or0123;

    endmodule
\end{minted}

\subsection {Four Input NOR Gate from Two Input NOR Gate}

\begin{minted} [mathescape] {verilog}
    module nor4input(input logic in0, in1, in2, in3,
        output logic out);

        wire nor01, nor23;
        wire or01, or23;

        nor #3 (nor01, in0, in1);
        nor #3 (or01, nor01, nor01);

        nor #3 (nor23, in2, in3);
        nor #3 (or23, nor23, nor23);

        nor #3 (out, or01, or23);

    endmodule
\end{minted}


\subsection {A 16 to 1 Multiplexer}

\begin{minted} {verilog}
    module mux16(input logic [15:0] in, input logic [3:0] sl,
        output logic out, muxlast);

        wire sl00, sl01, sl02, sl03; // original select lines
        wire sl10, sl11, sl12, sl13; // inverted select lines

        assign sl00 = sl[0];
        assign sl01 = sl[1];
        assign sl02 = sl[2];
        assign sl03 = sl[3];

        nor #3 (sl10, sl00, sl00);
        nor #3 (sl11, sl01, sl01);
        nor #3 (sl12, sl02, sl02);
        nor #3 (sl13, sl03, sl03);

        wire nor00, nor01, nor02, nor03, nor04, nor05, nor06, nor07,
            nor08, nor09, nor10, nor11, nor12, nor13, nor14, nor15;

        nor5input nor500 (.in0(sl03), .in1(sl02), .in2(sl01), .in3(sl00), .in4(in[00]),
        .out(nor00));
        nor5input nor501 (.in0(sl03), .in1(sl02), .in2(sl01), .in3(sl10), .in4(in[01]),
        .out(nor01));
        nor5input nor502 (.in0(sl03), .in1(sl02), .in2(sl11), .in3(sl00), .in4(in[02]),
        .out(nor02));
        nor5input nor503 (.in0(sl03), .in1(sl02), .in2(sl11), .in3(sl10), .in4(in[03]),
        .out(nor03));
        nor5input nor504 (.in0(sl03), .in1(sl12), .in2(sl01), .in3(sl00), .in4(in[04]),
        .out(nor04));
        nor5input nor505 (.in0(sl03), .in1(sl12), .in2(sl01), .in3(sl10), .in4(in[05]),
        .out(nor05));
        nor5input nor506 (.in0(sl03), .in1(sl12), .in2(sl11), .in3(sl00), .in4(in[06]),
        .out(nor06));
        nor5input nor507 (.in0(sl03), .in1(sl12), .in2(sl11), .in3(sl10), .in4(in[07]),
        .out(nor07));
        nor5input nor508 (.in0(sl13), .in1(sl02), .in2(sl01), .in3(sl00), .in4(in[08]),
        .out(nor08));
        nor5input nor509 (.in0(sl13), .in1(sl02), .in2(sl01), .in3(sl10), .in4(in[09]),
        .out(nor09));
        nor5input nor510 (.in0(sl13), .in1(sl02), .in2(sl11), .in3(sl00), .in4(in[10]),
        .out(nor10));
        nor5input nor511 (.in0(sl13), .in1(sl02), .in2(sl11), .in3(sl10), .in4(in[11]),
        .out(nor11));
        nor5input nor512 (.in0(sl13), .in1(sl12), .in2(sl01), .in3(sl00), .in4(in[12]),
        .out(nor12));
        nor5input nor513 (.in0(sl13), .in1(sl12), .in2(sl01), .in3(sl10), .in4(in[13]),
        .out(nor13));
        nor5input nor514 (.in0(sl13), .in1(sl12), .in2(sl11), .in3(sl00), .in4(in[14]),
        .out(nor14));
        nor5input nor515 (.in0(sl13), .in1(sl12), .in2(sl11), .in3(sl10), .in4(in[15]),
        .out(nor15), .f4high(muxlast));

        wire fnor0, fnor1, fnor2, fnor3;
        wire for0, for1, for2, for3;

        nor #3 (for0, fnor0, fnor0);
        nor #3 (for1, fnor1, fnor1);
        nor #3 (for2, fnor2, fnor2);
        nor #3 (for3, fnor3, fnor3);

        nor4input nor400 (.in0(nor00), .in1(nor01), .in2(nor02), .in3(nor03), .out(fnor0));
        nor4input nor401 (.in0(nor04), .in1(nor05), .in2(nor06), .in3(nor07), .out(fnor1));
        nor4input nor402 (.in0(nor08), .in1(nor09), .in2(nor10), .in3(nor11), .out(fnor2));
        nor4input nor403 (.in0(nor12), .in1(nor13), .in2(nor14), .in3(nor15), .out(fnor3));

        nor4input nor404 (.in0(for0), .in1(for1), .in2(for2), .in3(for3), .out(out));

    endmodule
\end{minted}

\subsection {An Abstracted Schematic of A 16 to 1 Multiplexer}

\begin{figure} [!h]
    \centering
    \resizebox{\textwidth}{!} {
        \includegraphics {tikzpics/epicmuxsixteen.pdf}
        %% where here: epicmuxsixteen
    }
    \caption{Schematic of a 16 to 1 MUX}
\end{figure}

\section {Edge Dectectors}

\subsection {Falling Edge Dectector}

\begin{minted} {verilog}
    module edge_detector(input logic clk,
        output logic feclk);

        /*
         * Falling edge detector
         * */

        wire notclk;

        nor #30 (notclk, clk, clk);
        nor #3 (feclk, notclk, clk);

    endmodule
\end{minted}

\subsection {Falling Edge Dectector with Inverted Output}

\begin{minted} {verilog}
    module edge_detector_inv(input logic clk,
        output logic notfeclk);

        /*
         * Falling edge detector,
         * with inverted output
         * */

        wire notclk;
        wire feclk;

        nor #30 (notclk, clk, clk);
        nor #3 (feclk, notclk, clk);
        nor #3 (notfeclk, feclk, feclk);

    endmodule
\end{minted}

\section {Logic Gates} \label{sec:logicgates}

\subsection {XOR Gate} \label{sec:xorgate}

\begin{figure} [!ht]
    \def\truthTableWidth{0.35\textwidth}
    \begin{tabularx} {\textwidth} {GT}
        \centering
        \vfill
        \resizebox{0.6\textwidth}{!} {
            \includegraphics {tikzpics/epicglvlxorgate.pdf}
            %% where here: epicglvlxorgate
        }
        \vfill
        &
        \vfill
        {\begin{tabularx} {0.75\linewidth} {*{3}{>{\centering\arraybackslash}X}}
            \toprule
            \textsc{\textbf{A}} & \textsc{\textbf{B}} & \textsc{\textbf{Y}} \\
            \midrule
            0   &   0   &   0   \\
            0   &   1   &   1   \\
            1   &   0   &   1   \\
            1   &   1   &   0   \\
            \bottomrule
        \end{tabularx}}
        \vfill
        \\
        \captionof{figure} {XOR Gate Using NOR Gates}
        \label{fig:glvlxorgate}
        &
        \captionof{table} {Truth Table of XOR Gate}
        \label{tbl:ttxorgate}
    \end{tabularx}
\end{figure}

\subsubsection{Verilog Implementation} \label{hdl:xorgate}

\begin{minted}[mathescape] {verilog}
    module xor_gate(input logic a, b,
        output logic y);

        wire anorb, nor1, nor2, xnor1;

        nor #3 (anorb, a, b);
        nor #3 (nor1, anorb, a);
        nor #3 (nor2, anorb, b);
        nor #3 (xnor1, nor1, nor2);
        nor #3 (y, xnor1, xnor1);

    endmodule
\end{minted}

\subsection {AND Gate} \label{sec:andgate}

\begin{figure} [!ht]
    \def\truthTableWidth{0.35\textwidth}
    \begin{tabularx} {\textwidth} {GT}
        \centering
        \vfill
        \resizebox{0.6\textwidth}{!} {
            \includegraphics {tikzpics/epicglvlandgate.pdf}
            %% where here: epicglvlandgate
        }
        \vfill
        &
        \vfill
        {\begin{tabularx} {0.75\linewidth} {*{3}{>{\centering\arraybackslash}X}}
            \toprule
            \textsc{\textbf{A}} & \textsc{\textbf{B}} & \textsc{\textbf{Y}} \\
            \midrule
            0   &   0   &   0   \\
            0   &   1   &   0   \\
            1   &   0   &   0   \\
            1   &   1   &   1   \\
            \bottomrule
        \end{tabularx}}
        \vfill
        \\
        \captionof{figure} {AND Gate Using NOR Gates}
        \label{fig:glvlandgate}
        &
        \captionof{table} {Truth Table of AND Gate}
        \label{tbl:ttandgate}
    \end{tabularx}
\end{figure}

\subsubsection{Verilog Implementation} \label{hdl:andgate}

\begin{minted}[mathescape] {verilog}
    module and_gate(input logic a, b,
        output logic y);

        wire anot, bnot;

        nor #3 (anot, a, a);
        nor #3 (bnot, b, b);
        nor #3 (y, anot, bnot);

    endmodule
\end{minted}

\subsection {OR Gate} \label{sec:orgate}

\begin{figure} [!ht]
    \def\truthTableWidth{0.35\textwidth}
    \begin{tabularx} {\textwidth} {GT}
        \centering
        \vfill
        \resizebox{0.6\textwidth}{!} {
            \includegraphics {tikzpics/epicglvlorgate.pdf}
            %% where here: epicglvlorgate
        }
        \vfill
        &
        \vfill
        {\begin{tabularx} {0.75\linewidth} {*{3}{>{\centering\arraybackslash}X}}
            \toprule
            \textsc{\textbf{A}} & \textsc{\textbf{B}} & \textsc{\textbf{Y}} \\
            \midrule
            0   &   0   &   0   \\
            0   &   1   &   1   \\
            1   &   0   &   1   \\
            1   &   1   &   1   \\
            \bottomrule
        \end{tabularx}}
        \vfill
        \\
        \captionof{figure} {OR Gate Using NOR Gates}
        \label{fig:glvlorgate}
        &
        \captionof{table} {Truth Table of OR Gate}
        \label{tbl:ttorgate}
    \end{tabularx}
\end{figure}

\subsubsection{Verilog Implementation} \label{hdl:orgate}

\begin{minted}[mathescape] {verilog}
    module or_gate(input logic a, b,
        output logic y);

        wire anorb;

        nor #3 (anorb, a, b);
        nor #3 (y, anorb, anorb);

    endmodule
\end{minted}

\subsection {NOT Gate} \label{sec:notgate}

\begin{figure} [!ht]
    \def\truthTableWidth{0.35\textwidth}
    \begin{tabularx} {\textwidth} {GT}
        \centering
        \vfill
        \resizebox{0.6\textwidth}{!} {
            \includegraphics {tikzpics/epicglvlnotgate.pdf}
            %% where here: epicglvlnotgate
        }
        \vfill
        &
        \vfill
        {\begin{tabularx} {0.75\linewidth} {*{2}{>{\centering\arraybackslash}X}}
            \toprule
            \textsc{\textbf{A}} & \textsc{\textbf{Y}} \\
            \midrule
            0   &   1   \\
            1   &   0   \\
            \bottomrule
        \end{tabularx}}
        \vfill
        \\
        \captionof{figure} {NOT Gate Using NOR Gate}
        \label{fig:glvlnotgate}
        &
        \captionof{table} {Truth Table of NOT Gate}
        \label{tbl:ttnotgate}
    \end{tabularx}
\end{figure}

\subsubsection{Verilog Implementation} \label{hdl:notgate}

\begin{minted}[mathescape] {verilog}
    module not_gate(input logic a,
        output logic y);

        nor #3 (y, a, a);

    endmodule
\end{minted}

\subsection {NOR Gate} \label{sec:norgate}

\begin{figure} [!ht]
    \def\truthTableWidth{0.35\textwidth}
    \begin{tabularx} {\textwidth} {GT}
        \centering
        \vfill
        \resizebox{0.6\textwidth}{!} {
            \includegraphics {tikzpics/epicglvlnorgate.pdf}
            %% where here: epicglvlnorgate
        }
        \vfill
        &
        \vfill
        {\begin{tabularx} {0.75\linewidth} {*{3}{>{\centering\arraybackslash}X}}
            \toprule
            \textsc{\textbf{A}} & \textsc{\textbf{B}} & \textsc{\textbf{Y}} \\
            \midrule
            0   &   0   &   1   \\
            0   &   1   &   0   \\
            1   &   0   &   0   \\
            1   &   1   &   0   \\
            \bottomrule
        \end{tabularx}}
        \vfill
        \\
        \captionof{figure} {NOR Gate Using NOR Gate}
        \label{fig:glvlnorgate}
        &
        \captionof{table} {Truth Table of NOR Gate}
        \label{tbl:ttnorgate}
    \end{tabularx}
\end{figure}

\subsubsection{Verilog Implementation} \label{hdl:norgate}

\begin{minted}[mathescape] {verilog}
    module nor_gate(input logic a, b,
        output logic y);

        nor #3 (y, a, b);

    endmodule
\end{minted}

\subsection {NAND Gate} \label{sec:nandgate}

\begin{figure} [!ht]
    \def\truthTableWidth{0.35\textwidth}
    \begin{tabularx} {\textwidth} {GT}
        \centering
        \vfill
        \resizebox{0.6\textwidth}{!} {
            \includegraphics {tikzpics/epicglvlnandgate.pdf}
            %% where here: epicglvlnandgate
        }
        \vfill
        &
        \vfill
        {\begin{tabularx} {0.75\linewidth} {*{3}{>{\centering\arraybackslash}X}}
            \toprule
            \textsc{\textbf{A}} & \textsc{\textbf{B}} & \textsc{\textbf{Y}} \\
            \midrule
            0   &   0   &   1   \\
            0   &   1   &   1   \\
            1   &   0   &   1   \\
            1   &   1   &   0   \\
            \bottomrule
        \end{tabularx}}
        \vfill
        \\
        \captionof{figure} {NAND Gate Using NOR Gates}
        \label{fig:glvlnandgate}
        &
        \captionof{table} {Truth Table of NAND Gate}
        \label{tbl:ttnandgate}
    \end{tabularx}
\end{figure}

\subsubsection{Verilog Implementation} \label{hdl:nandgate}

\begin{minted}[mathescape] {verilog}
    module nand_gate(input logic a, b,
        output logic y);

        wire nota, notb;
        wire andab;

        nor #3 (nota, a, a);
        nor #3 (notb, b, b);
        nor #3 (andab, nota, notb);
        nor #3 (y, andab, andab);

    endmodule

\end{minted}

\section {Full Adder}

\begin{minted}[mathescape] {verilog}
    module full_adder(input logic a, b, cin,
        output logic sum, cout);

        wire anorb;
        wire xnor1, xnor2;
        wire xnor1nor1, xnor1nor2, xnor2nor1, xnor2nor2;
        wire xnor1norcin;

        // sum
        nor #3 (anorb, a, b);
        nor #3 (xnor1nor1, a, anorb);
        nor #3 (xnor1nor2, b, anorb);
        nor #3 (xnor1, xnor1nor1, xnor1nor2);

        nor #3 (xnor1norcin, xnor1, cin);
        nor #3 (xnor2nor1, xnor1, xnor1norcin);
        nor #3 (xnor2nor2, cin, xnor1norcin);
        nor #3 (sum, xnor2nor1, xnor2nor2);

        // cout
        nor #3 (cout, xnor1norcin, anorb);

    endmodule
\end{minted}

\end{document}
