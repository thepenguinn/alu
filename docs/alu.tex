\documentclass[a4paper, 10pt]{article}

\usepackage[siunitx]{circuitikz}
\usepackage[margin = 1in]{geometry}
\usepackage{rotating}
\usepackage{ifthen}

\usetikzlibrary{intersections}
\usetikzlibrary{calc}

\def\cornerhypo{0.15}
\def\cornerbase{0.5cm}
\def\cV{|-}
\def\cH{-|}

% to get rid of lines not properly joining after setting
% a coordinate at some other position and continue drawing
\newcommand\slightleft[0] {
    -- +(-0.005,0)
}

\newcommand\slightright[0] {
    -- +(0.005,0)
}

\newcommand\slightup[0] {
    -- +(0,0.005)
}

\newcommand\slightdown[0] {
    -- +(0,-0.005)
}

\newcommand\markconstcorner[2] {
    coordinate (LCJTMP)
    #2 coordinate (finaltmp)

    (LCJTMP#1finaltmp) coordinate (LCVTMP)
    ($(LCVTMP)!\cornerbase!0:(finaltmp)$) coordinate (SCBTMP)
    ($(LCVTMP)!\cornerbase!0:(LCJTMP)$) coordinate (FCBTMP)

}

\newcommand\constcorner[2] {
    \markconstcorner {#1} {#2}

    (LCJTMP) -- (FCBTMP) -- (SCBTMP)
}

\newcommand\constcornerright[1] {
    \slightright
    \constcorner {\cH} {#1}
}

\newcommand\constcornerleft[1] {
    \slightleft
    \constcorner {\cH} {#1}
}

\newcommand\constcornerup [1] {
    \slightup
    \constcorner {\cV} {#1}
}

\newcommand\constcornerdown [1] {
    \slightdown
    \constcorner {\cV} {#1}
}

%%
\newcommand\constcornerhoz [1] {
    \constcorner {\cH} {#1}
}

\newcommand\constcornervrt [1] {
    \constcorner {\cV} {#1}
}

\newcommand\markconstcornerfrom[2] {

    \markconstcorner {#1} {#2}

    (LCVTMP) coordinate (fsttmp)
    ($(LCVTMP)!\cornerbase!180:(FCBTMP)$) coordinate (LCVTMP)
    (fsttmp) coordinate (FCBTMP)

    (SCBTMP) ++($(LCVTMP)-(FCBTMP)$) coordinate (SCBTMP)
}

\newcommand\constcornerfrom[2] {

    \markconstcornerfrom {#1} {#2}
    (LCJTMP) -- (FCBTMP) -- (SCBTMP)
}

\newcommand\constcornerfromright[1] {
    \slightright
    \constcornerfrom {\cH} {#1}
}

\newcommand\constcornerfromleft[1] {
    \slightleft
    \constcornerfrom {\cH} {#1}
}

\newcommand\constcornerfromup [1] {
    \slightup
    \constcornerfrom {\cV} {#1}
}

\newcommand\constcornerfromdown [1] {
    \slightdown
    \constcornerfrom {\cV} {#1}
}

\newcommand\markvarcornerpos[2] {
    coordinate (LCJTMP)
    #2 coordinate (SCBTMP)

    (LCJTMP#1SCBTMP) coordinate (LCVTMP)
    ($(SCBTMP)!0.1cm!-45:(LCVTMP)$) coordinate (cornedgtmp)
    (intersection of LCJTMP--LCVTMP and SCBTMP--cornedgtmp)
    coordinate (FCBTMP)
}

\newcommand\markvarcornerneg[2] {
    coordinate (LCJTMP)
    #2 coordinate (SCBTMP)

    (LCJTMP#1SCBTMP) coordinate (LCVTMP)
    ($(SCBTMP)!0.1cm!+45:(LCVTMP)$) coordinate (cornedgtmp)
    (intersection of LCJTMP--LCVTMP and SCBTMP--cornedgtmp)
    coordinate (FCBTMP)
}

\newcommand\varcornerpos[2] {

    \markvarcornerpos {#1} {#2}
    (LCJTMP) -- (FCBTMP) -- (SCBTMP)
}

\newcommand\varcornerneg[2] {

    \markvarcornerneg {#1} {#2}
    (LCJTMP) -- (FCBTMP) -- (SCBTMP)

}

\newcommand\varcornerupright[1] {
    \slightup
    \varcornerneg {\cV} {#1}
}
\newcommand\varcornerdownleft[1] {
    \slightdown
    \varcornerneg {\cV} {#1}
}

\newcommand\varcornerupleft[1] {
    \slightup
    \varcornerpos {\cV} {#1}
}
\newcommand\varcornerdownright[1] {
    \slightdown
    \varcornerpos {\cV} {#1}
}

\newcommand\varcornerrightdown[1] {
    \slightright
    \varcornerneg {\cH} {#1}
}
\newcommand\varcornerleftup[1] {
    \slightleft
    \varcornerneg {\cH} {#1}
}

\newcommand\varcornerrightup[1] {
    \slightright
    \varcornerpos {\cH} {#1}
}
\newcommand\varcornerleftdown[1] {
    \slightleft
    \varcornerpos {\cH} {#1}
}

%%
\newcommand\varcornervrtneg[1] {
    \varcornerneg {\cV} {#1}
}

\newcommand\varcornervrtpos[1] {
    \varcornerpos {\cV} {#1}
}

\newcommand\varcornerhozneg[1] {
    \varcornerneg {\cH} {#1}
}

\newcommand\varcornerhozpos[1] {
    \varcornerpos {\cH} {#1}
}

% AROUND CORNER

\newcommand\markaroundcornerpos[3] {
    \markvarcornerpos {#2} {#3}

    ($(SCBTMP)!#1!90:(FCBTMP)$) coordinate (bevltmp)
    ($(SCBTMP)!#1!90:(LCVTMP)$) coordinate (nexsidetmp)
    ($(bevltmp)!0.1cm!90:(SCBTMP)$) coordinate (bevlpartmp)

    (LCJTMP#2nexsidetmp) coordinate (LCVTMP)

    (intersection of nexsidetmp--LCVTMP and bevltmp--bevlpartmp)
    coordinate (SCBTMP)

    (intersection of LCJTMP--LCVTMP and bevltmp--bevlpartmp)
    coordinate (FCBTMP)

}

\newcommand\markaroundcornerneg[3] {
    \markvarcornerneg {#2} {#3}

    ($(SCBTMP)!#1!-90:(FCBTMP)$) coordinate (bevltmp)
    ($(SCBTMP)!#1!-90:(LCVTMP)$) coordinate (nexsidetmp)
    ($(bevltmp)!0.1cm!-90:(SCBTMP)$) coordinate (bevlpartmp)

    (LCJTMP#2nexsidetmp) coordinate (LCVTMP)

    (intersection of nexsidetmp--LCVTMP and bevltmp--bevlpartmp)
    coordinate (SCBTMP)

    (intersection of LCJTMP--LCVTMP and bevltmp--bevlpartmp)
    coordinate (FCBTMP)

}

\newcommand\aroundcornerpos[3] {

    let \p{MAGCORDTMP} = #1,
    \n{MAGTMP} = {veclen(\x{MAGCORDTMP}, \y{MAGCORDTMP})}
    in
    \markaroundcornerpos {\n{MAGTMP}} {#2} {#3}
    (LCJTMP) -- (FCBTMP) -- (SCBTMP)
}

\newcommand\aroundcornerneg[3] {

    let \p{MAGCORDTMP} = #1,
    \n{MAGTMP} = {veclen(\x{MAGCORDTMP}, \y{MAGCORDTMP})}
    in
    \markaroundcornerneg {\n{MAGTMP}} {#2} {#3}
    (LCJTMP) -- (FCBTMP) -- (SCBTMP)

}

\newcommand\aroundcornerupright[2] {
    \slightup
    \aroundcornerneg {#1} {\cV} {#2}
}
\newcommand\aroundcornerdownleft[2] {
    \slightdown
    \aroundcornerneg {#1} {\cV} {#2}
}

\newcommand\aroundcornerupleft[2] {
    \slightup
    \aroundcornerpos {#1} {\cV} {#2}
}
\newcommand\aroundcornerdownright[2] {
    \slightdown
    \aroundcornerpos {#1} {\cV} {#2}
}

\newcommand\aroundcornerrightdown[2] {
    \slightright
    \aroundcornerneg {#1} {\cH} {#2}
}
\newcommand\aroundcornerleftup[2] {
    \slightleft
    \aroundcornerneg {#1} {\cH} {#2}
}

\newcommand\aroundcornerrightup[2] {
    \slightright
    \aroundcornerpos {#1} {\cH} {#2}
}
\newcommand\aroundcornerleftdown[2] {
    \slightleft
    \aroundcornerpos {#1} {\cH} {#2}
}

% AROUND CORNER TILL
% gap vrt/hoz pivotpoint tillpoint vectorfromtillpoint

\newcommand\markaroundcornertillpos[5] {

    \markaroundcornerpos {#1} {#2} {#3}
    (LCJTMP)
    #4 coordinate (TILLTMP)
    +#5 coordinate (TILLPARTMP)

    (intersection of TILLTMP--TILLPARTMP and FCBTMP--SCBTMP)
    coordinate (SCBTMP)
    (FCBTMP#2SCBTMP) coordinate (LCVTMP)

}

\newcommand\markaroundcornertillneg[5] {

    \markaroundcornerneg {#1} {#2} {#3}
    (LCJTMP)
    #4 coordinate (TILLTMP)
    +#5 coordinate (TILLPARTMP)

    (intersection of TILLTMP--TILLPARTMP and FCBTMP--SCBTMP)
    coordinate (SCBTMP)
    (FCBTMP#2SCBTMP) coordinate (LCVTMP)

}

\newcommand\aroundcornertillpos[5] {

    let \p{MAGCORDTMP} = #1,
    \n{MAGTMP} = {veclen(\x{MAGCORDTMP}, \y{MAGCORDTMP})}
    in
    \markaroundcornertillpos {\n{MAGTMP}} {#2} {#3} {#4} {#5}
    (LCJTMP) -- (FCBTMP) -- (SCBTMP)
}

\newcommand\aroundcornertillneg[5] {

    let \p{MAGCORDTMP} = #1,
    \n{MAGTMP} = {veclen(\x{MAGCORDTMP}, \y{MAGCORDTMP})}
    in
    \markaroundcornertillneg {\n{MAGTMP}} {#2} {#3} {#4} {#5}
    (LCJTMP) -- (FCBTMP) -- (SCBTMP)

}

\newcommand\aroundcornertillupright[4] {
    \slightup
    \aroundcornertillneg {#1} {\cV} {#2} {#3} {#4}
}
\newcommand\aroundcornertilldownleft[4] {
    \slightdown
    \aroundcornertillneg {#1} {\cV} {#2} {#3} {#4}
}

\newcommand\aroundcornertillupleft[4] {
    \slightup
    \aroundcornertillpos {#1} {\cV} {#2} {#3} {#4}
}
\newcommand\aroundcornertilldownright[4] {
    \slightdown
    \aroundcornertillpos {#1} {\cV} {#2} {#3} {#4}
}

\newcommand\aroundcornertillrightdown[4] {
    \slightright
    \aroundcornertillneg {#1} {\cH} {#2} {#3} {#4}
}
\newcommand\aroundcornertillleftup[4] {
    \slightleft
    \aroundcornertillneg {#1} {\cH} {#2} {#3} {#4}
}

\newcommand\aroundcornertillrightup[4] {
    \slightright
    \aroundcornertillpos {#1} {\cH} {#2} {#3} {#4}
}
\newcommand\aroundcornertillleftdown[4] {
    \slightleft
    \aroundcornertillpos {#1} {\cH} {#2} {#3} {#4}
}

% UNDER CORNER

\newcommand\markundercornerpos[3] {
    \markvarcornerpos {#2} {#3}

    ($(SCBTMP)!#1!-90:(FCBTMP)$) coordinate (bevltmp)
    ($(SCBTMP)!#1!-90:(LCVTMP)$) coordinate (nexsidetmp)
    ($(bevltmp)!0.1cm!-90:(SCBTMP)$) coordinate (bevlpartmp)

    (LCJTMP#2nexsidetmp) coordinate (LCVTMP)

    (intersection of nexsidetmp--LCVTMP and bevltmp--bevlpartmp)
    coordinate (SCBTMP)

    (intersection of LCJTMP--LCVTMP and bevltmp--bevlpartmp)
    coordinate (FCBTMP)

}

\newcommand\markundercornerneg[3] {
    \markvarcornerneg {#2} {#3}

    ($(SCBTMP)!#1!90:(FCBTMP)$) coordinate (bevltmp)
    ($(SCBTMP)!#1!90:(LCVTMP)$) coordinate (nexsidetmp)
    ($(bevltmp)!0.1cm!90:(SCBTMP)$) coordinate (bevlpartmp)

    (LCJTMP#2nexsidetmp) coordinate (LCVTMP)

    (intersection of nexsidetmp--LCVTMP and bevltmp--bevlpartmp)
    coordinate (SCBTMP)

    (intersection of LCJTMP--LCVTMP and bevltmp--bevlpartmp)
    coordinate (FCBTMP)

}

\newcommand\undercornerpos[3] {

    let \p{MAGCORDTMP} = #1,
    \n{MAGTMP} = {veclen(\x{MAGCORDTMP}, \y{MAGCORDTMP})}
    in
    \markundercornerpos {\n{MAGTMP}} {#2} {#3}
    (LCJTMP) -- (FCBTMP) -- (SCBTMP)
}

\newcommand\undercornerneg[3] {

    let \p{MAGCORDTMP} = #1,
    \n{MAGTMP} = {veclen(\x{MAGCORDTMP}, \y{MAGCORDTMP})}
    in
    \markundercornerneg {\n{MAGTMP}} {#2} {#3}
    (LCJTMP) -- (FCBTMP) -- (SCBTMP)

}

\newcommand\undercornerupright[2] {
    \slightup
    \undercornerneg {#1} {\cV} {#2}
}
\newcommand\undercornerdownleft[2] {
    \slightdown
    \undercornerneg {#1} {\cV} {#2}
}

\newcommand\undercornerupleft[2] {
    \slightup
    \undercornerpos {#1} {\cV} {#2}
}
\newcommand\undercornerdownright[2] {
    \slightdown
    \undercornerpos {#1} {\cV} {#2}
}

\newcommand\undercornerrightdown[2] {
    \slightright
    \undercornerneg {#1} {\cH} {#2}
}
\newcommand\undercornerleftup[2] {
    \slightleft
    \undercornerneg {#1} {\cH} {#2}
}

\newcommand\undercornerrightup[2] {
    \slightright
    \undercornerpos {#1} {\cH} {#2}
}
\newcommand\undercornerleftdown[2] {
    \slightleft
    \undercornerpos {#1} {\cH} {#2}
}

% UNDER CORNER TILL

\newcommand\markundercornertillpos[5] {

    \markundercornerpos {#1} {#2} {#3}
    (LCJTMP)
    #4 coordinate (TILLTMP)
    +#5 coordinate (TILLPARTMP)

    (intersection of TILLTMP--TILLPARTMP and FCBTMP--SCBTMP)
    coordinate (SCBTMP)
    (FCBTMP#2SCBTMP) coordinate (LCVTMP)

}

\newcommand\markundercornertillneg[5] {

    \markundercornerneg {#1} {#2} {#3}
    (LCJTMP)
    #4 coordinate (TILLTMP)
    +#5 coordinate (TILLPARTMP)

    (intersection of TILLTMP--TILLPARTMP and FCBTMP--SCBTMP)
    coordinate (SCBTMP)
    (FCBTMP#2SCBTMP) coordinate (LCVTMP)

}

\newcommand\undercornertillpos[5] {

    let \p{MAGCORDTMP} = #1,
    \n{MAGTMP} = {veclen(\x{MAGCORDTMP}, \y{MAGCORDTMP})}
    in
    \markundercornertillpos {\n{MAGTMP}} {#2} {#3} {#4} {#5}
    (LCJTMP) -- (FCBTMP) -- (SCBTMP)
}

\newcommand\undercornertillneg[5] {

    let \p{MAGCORDTMP} = #1,
    \n{MAGTMP} = {veclen(\x{MAGCORDTMP}, \y{MAGCORDTMP})}
    in
    \markundercornertillneg {\n{MAGTMP}} {#2} {#3} {#4} {#5}
    (LCJTMP) -- (FCBTMP) -- (SCBTMP)

}

\newcommand\undercornertillupright[4] {
    \slightup
    \undercornertillneg {#1} {\cV} {#2} {#3} {#4}
}
\newcommand\undercornertilldownleft[4] {
    \slightdown
    \undercornertillneg {#1} {\cV} {#2} {#3} {#4}
}

\newcommand\undercornertillupleft[4] {
    \slightup
    \undercornertillpos {#1} {\cV} {#2} {#3} {#4}
}
\newcommand\undercornertilldownright[4] {
    \slightdown
    \undercornertillpos {#1} {\cV} {#2} {#3} {#4}
}

\newcommand\undercornertillrightdown[4] {
    \slightright
    \undercornertillneg {#1} {\cH} {#2} {#3} {#4}
}
\newcommand\undercornertillleftup[4] {
    \slightleft
    \undercornertillneg {#1} {\cH} {#2} {#3} {#4}
}

\newcommand\undercornertillrightup[4] {
    \slightright
    \undercornertillpos {#1} {\cH} {#2} {#3} {#4}
}
\newcommand\undercornertillleftdown[4] {
    \slightleft
    \undercornertillpos {#1} {\cH} {#2} {#3} {#4}
}

% \ctikzsubcircuitdef{msFlop} {EC, IEC, D, RST, Q} {
% \ctikzsubcircuitdef{test} {EC} {
%     coordinate(#1-EC)
%     \corneranticlk {++(0,4)}
%     \corneranticlk {++(4,0)}
%     -- ++(4,0)
% }

\tikzset{fullAdder/.style = {muxdemux, muxdemux def = {
    NL = 3, NR = 3, NB = 0, NT = 0,
    Lh = 3, Rh = 3,
    w = 5 },
draw only right pins = {1, 3} }
}

\tikzset{aluInit/.style = {muxdemux, muxdemux def = {
    NL = 2, NR = 2, NB = 6, NT = 2,
    Lh = 3, Rh = 3,
    % Lh = 5.35, Rh = 3,
    w = 5 },
draw only top pins = {1},
draw only bottom pins = {5, 6}, }
}


\tikzset{logicOp/.style = {muxdemux, muxdemux def = {
    NL = 3, NR = 2, NB = 0, NT = 0,
    Lh = 2.5, Rh = 2.5,
    w = 4
    },
    draw only left pins = {2,3},
    draw only right pins = {1},
    }
}

\tikzset{msFlipFlop/.style = {muxdemux, muxdemux def = {
    NL = 2, NR = 2, NB = 2, NT = 6,
    Lh = 3, Rh = 3,
    w = 5
    },
    draw only left pins = {2},
    draw only top pins = {1,2},
    draw only bottom pins = {2},
    }
}

\tikzset{mux08/.style = {muxdemux, muxdemux def = {
    NL = 10, NR = 3, NB = 5, NT = 0,
    % Lh = 13, Rh = 7.8,
    Lh = 10, Rh = 6.5,
    % inset Lh = 4.5, inset Rh = 2.75,
    inset Lh = 3.5, inset Rh = 1.75,
    inset w = 1,
    % w = 4.5,
    w = 3.25,
    square pins = 1
    },
    draw only left pins={2-5,6-9},
    draw only right pins={1-2},
    draw only bottom pins={2-4}
    }
}

%  node [muxdemux, rotate = -90, muxdemux def = {NL=10, NR=3, NT=0, NB=4,
%  w = 2.5, Lh = 10, Rh = 6.5, inset Lh = 3.5, inset w = .75, inset Rh = 1.75,
%  square pins=1
%  },
%  draw only left pins={2-9},
%  draw only right pins={1-2},
%  draw only bottom pins={2-4}
%  ]
%  (mux) {
%      \begin{turn}{90}
%          \texttt{MUX8}
%      \end{turn}
%  }

%  % (mux.rpin 3) -- ++(1,1)

\def\fulladdername{\texttt{ADDER}}
% \def\logicopstyle{\texttt{}}


% \ctikzsubcircuitdef{onebitalu} {ina, inb, eclk, ieclk, rst, op0, op1, op2, regout, aluout} {
\ctikzsubcircuitdef{onebitalu} {L} {
    coordinate (#1-L)
    node [fullAdder] (#1-flad) {\fulladdername}
}

% #1 pinname pinnumber
\newcommand\markcoordinate[3] {

    coordinate (#1-#2) at (#1.#3)
    coordinate (#1-b#2) at (#1.b#3)
    ($(#1-#2)!0.25cm!180:(#1-b#2)$)
    coordinate (#1-p#2)

}

\ctikzsubcircuitdef{subctkfullAdder} {
    A, B, Cin, Cout, Sum,
    bA, bB, bCin, bCout, bSum%
} {
    node (#1) [fullAdder, anchor = lpin 1]{}%
    \markcoordinate {#1} {B}    {lpin 1}
    \markcoordinate {#1} {A}    {lpin 3}
    \markcoordinate {#1} {Cin}  {lpin 2}
    \markcoordinate {#1} {Sum}  {rpin 3}
    \markcoordinate {#1} {Cout} {rpin 1}%
}

\ctikzsubcircuitdef{subctkaluInit} {
    B, Cin, Op, Mux, Rst, Bout, Cout,
    bB, bCin, bOp, bMux, bRst, bBout, bCout%
} {
    node (#1) [aluInit, anchor = lpin 1]{}%
    \markcoordinate {#1} {B}    {lpin 1}
    \markcoordinate {#1} {Cin}  {lpin 2}
    \markcoordinate {#1} {Bout} {rpin 1}
    \markcoordinate {#1} {Cout} {rpin 2}
    \markcoordinate {#1} {Mux}  {bpin 6}
    \markcoordinate {#1} {Op}   {bpin 5}
    \markcoordinate {#1} {Rst}  {tpin 1}%
}

\ctikzsubcircuitdef{subctkmsFlipFlop} {
    D, Rst, Eclk, Ieclk, Q, Qbar,
    bD, bRst, bEclk, bIeclk, bQ, bQbar%
} {
    node (#1) [msFlipFlop, anchor = lpin 1]{}%
    \markcoordinate {#1} {D}     {lpin 2}
    \markcoordinate {#1} {Rst}   {bpin 2}
    \markcoordinate {#1} {Eclk}  {tpin 1}
    \markcoordinate {#1} {Ieclk} {tpin 2}
    \markcoordinate {#1} {Q}     {rpin 2}
    \markcoordinate {#1} {Qbar}  {rpin 1}%
}


\ctikzsubcircuitdef{subctklogicOp} {
    Ain, Bin, Out,
    bAin, bBin, bOut%
} {
    node (#1) [logicOp, anchor = lpin 1]{}%
    \markcoordinate {#1} {Ain} {lpin 2}
    \markcoordinate {#1} {Bin} {lpin 3}
    \markcoordinate {#1} {Out} {rpin 1}%
}

\ctikzsubcircuitdef{subctkmuxEight} {
    in0, in2, in3, in4, in5, in6, in7, sl0, sl1, sl2, out, and01,
    bin0, bin2, bin3, bin4, bin5, bin6, bin7, bsl0, bsl1, bsl2, bout, band01,
    pin0, pin2, pin3, pin4, pin5, pin6, pin7, psl0, psl1, psl2, pout, pand01%
} {
    node (#1) [mux08, anchor = lpin 1] {}%
    \markcoordinate {#1}   {in0} {lpin 2}
    \markcoordinate {#1}   {in2} {lpin 4}
    \markcoordinate {#1}   {in3} {lpin 5}
    \markcoordinate {#1}   {in4} {lpin 6}
    \markcoordinate {#1}   {in5} {lpin 7}
    \markcoordinate {#1}   {in6} {lpin 8}
    \markcoordinate {#1}   {in7} {lpin 9}%
    \markcoordinate {#1}   {sl0} {bpin 4}
    \markcoordinate {#1}   {sl1} {bpin 3}
    \markcoordinate {#1}   {sl2} {bpin 2}%
    \markcoordinate {#1} {and01} {rpin 1}
    \markcoordinate {#1}   {out} {rpin 2}
}

\ctikzsubcircuitactivate{test}
\ctikzsubcircuitactivate{onebitalu}
\ctikzsubcircuitactivate{subctkfullAdder}
\ctikzsubcircuitactivate{subctkaluInit}
\ctikzsubcircuitactivate{subctkmsFlipFlop}
\ctikzsubcircuitactivate{subctklogicOp}
\ctikzsubcircuitactivate{subctkmuxEight}

\newcommand\fullAdder[3] {
    \subctkfullAdder{#1} {#2}
    (#1.center) node[]{#3}
}

\newcommand\aluInit[3] {
    \subctkaluInit{#1} {#2}
    (#1.center) node[]{#3}
}

\newcommand\msFlipFlop[3] {
    \subctkmsFlipFlop{#1} {#2}
    (#1.center) node[]{#3}
}

\newcommand\logicOp[3] {
    \subctklogicOp{#1} {#2}
    (#1.center) node[]{#3}
}

\newcommand\muxEight[3] {
    \subctkmuxEight{#1} {#2}
    (#1.narrow center) node[]{#3}
}


% \newcommand\testing[] {
%
% }

\begin{document}

\section{1 bit ALU}

%  % \ctikzset{muxdemuxes/thickness = 3, multipoles/external pins thickness = 1.5, bipoles/tline/width = 3}

%  \draw[semithick]

%  node [muxdemux, rotate = -90, muxdemux def = {NL=10, NR=3, NT=0, NB=4,
%  w = 2.5, Lh = 10, Rh = 6.5, inset Lh = 3.5, inset w = .75, inset Rh = 1.75,
%  square pins=1
%  },
%  draw only left pins={2-9},
%  draw only right pins={1-2},
%  draw only bottom pins={2-4}]
%  (mux) {
%      \begin{turn}{90}
%          \texttt{MUX8}
%      \end{turn}
%  }

%  % (mux.rpin 3) -- ++(1,1)
%  ;

% \tikzset{blockdef-above/.style={
%     {Straight Barb[harpoon, right, length=-0.2cm]}-{Straight Barb[harpoon, left, length=-0.2cm]},
%     black,
%     }}

% \tikzset{barbarrow/.style = {> = {Straight Barb[right, length = 5pt, width = 5pt]}}}

\begin{figure}[!h]
    \centering
    \resizebox{\textwidth}{!}{
        \begin{circuitikz}[american]

            \draw [line join = round]

            (0,0)

            \fullAdder{flad}{Cin} {\texttt{\fulladdername}}
            (flad-Cout)
            \constcornerhoz {++(0.75,1)} -- ++(0.25,0)

            \aluInit{ainit}{Cin} {\texttt{ALU INIT}}
            (ainit-Cout)
            \constcornerhoz {++(0.75,-1)} -- ++(0.25,0)

            \msFlipFlop{msff}{D} {\texttt{MS FLIPFLOP}}

            (ainit-Bout) ++(0,2)
            coordinate (ainitBoutheight)

            (ainit-Bout)
            \constcornerright {++(0.75,1)}

            \constcornerup {(ainitBoutheight)}
            \constcornerfromleft {(flad-pB)}
            coordinate (fladleftvrt)
            \constcornerdown {(flad-B)}
            -- (flad-B)

            (msff-Q) -- (msff-pQ) -- ++(0.5,0)
            coordinate (msffQout)
            -- (ainitBoutheight-|msffQout)
            \constcornerup {++($(flad-B)-(flad-Cin)$) ++(-1,0)}

            \aroundcornerleftdown {($(flad-B)-(flad-Cin)$)} {(fladleftvrt)}
            \aroundcornertilldownright {($(flad-B)-(flad-Cin)$)} {(flad-pB)}
            {(flad-Cin)} {(1,0)}
            -- (flad-Cin)

            (msff-Rst)
            \constcornerdown {++(1,-0.75)}
            \constcornerright {++(2,-1)}
            coordinate (msffRstvrt)
            \constcornerdown {++(-1,-3)}
            -- ++(-0.5,0)

            coordinate (tmp)
            \muxEight {mux} {pand01} {\texttt{MUX}}
            (tmp)
            -- (mux-and01)

            ($(mux-in5)!0.5!(mux-in6)$)
            coordinate (muxbotmid)
            (muxbotmid-|ainit-Mux) ++(-1,0)
            coordinate (muxbotcnt)

            (muxbotcnt) ++(0,0.75)
            \logicOp {orgate} {Out} {\texttt{OR}}

            (muxbotcnt) ++(0,-2)
            \logicOp {nandgate} {Out} {\texttt{NAND}}

            (muxbotcnt-|orgate-Ain) ++(0,2)
            \logicOp {notgate} {Out} {\texttt{NOT}}

            (muxbotcnt-|orgate-Ain) ++(0,-0.75)
            \logicOp {norgate} {Out} {\texttt{NOR}}

            % (notgate-Out) -- (notgate-Out-|muxbotcnt)
            % (orgate-Out) -- (orgate-Out-|muxbotcnt)
            % (norgate-Out) -- (norgate-Out-|muxbotcnt)
            % (nandgate-Out) -- (nandgate-Out-|muxbotcnt)

            (muxbotcnt) ++(1.5,0)
            coordinate (muxbotmid)

            (orgate-Out)
            \varcornerrightdown {(muxbotmid|-mux-in5)}
            -- (mux-in5)

            (norgate-Out)
            \varcornerrightup {(muxbotmid|-mux-in6)}
            -- (mux-in6)

            (notgate-Out)
            \aroundcornertillrightdown
            {($(mux-in5)-(mux-in6)$)} {(muxbotmid|-mux-in5)}
            {(mux-in4)} {(1,0)}
            -- (mux-in4)

            (nandgate-Out)
            \aroundcornertillrightup
            {($(mux-in5)-(mux-in6)$)} {(muxbotmid|-mux-in6)}
            {(mux-in7)} {(1,0)}
            -- (mux-in7)

            (orgate-Out) ++(0,3)
            \logicOp {andgate} {Out} {\texttt{AND}}

            (notgate-Out) ++(0,3)
            \logicOp {xorgate} {Out} {\texttt{XOR}}

            (andgate-Out)
            \varcornerrightdown {(muxbotmid|-mux-in3)}
            -- (mux-in3)

            (xorgate-Out)
            \aroundcornertillrightdown
            {($(mux-in5)-(mux-in6)$)} {(muxbotmid|-mux-in3)}
            {(mux-in2)} {(1,0)}
            -- (mux-in2)

            (flad-Sum)
            \aroundcornertillrightdown
            {($(mux-in0)-(mux-in3)$)} {(muxbotmid|-mux-in3)}
            {(mux-in0)} {(1,0)}
            -- (mux-in0)

            (ainit-Mux)
            \varcornerdownright {(msff-D|-msffRstvrt) ++(0,-0.5)}
            coordinate (ainitMuxbend)
            -- (ainitMuxbend-|msffRstvrt)

            (ainit-Op)
            \aroundcornerdownright
            {($(ainit-Op)-(ainit-Mux)$)} {(ainitMuxbend)}
            \constcornerright {++(6,-1)}
            \constcornerfromdown {(mux-sl0)}
            coordinate (tmp)
            -- (tmp-|mux-sl0)

            % A input

            (xorgate-Ain)
            -- ++(-2,0)
            coordinate (Ainbrvrt)

            (andgate-Ain) -- (andgate-Ain-|Ainbrvrt)
            (orgate-Ain)  --  (orgate-Ain-|Ainbrvrt)
            (notgate-Ain) -- (notgate-Ain-|Ainbrvrt)
            (norgate-Ain) -- (norgate-Ain-|Ainbrvrt)

            (nandgate-Ain)
            \constcornerleft {(nandgate-Ain-|Ainbrvrt)}
            -- (flad-A-|Ainbrvrt)
            -- (flad-A)

            % B input

            (xorgate-Bin)
            -- ++(-3,0)
            coordinate (Binbrvrt)

            (andgate-Bin) -- (andgate-Bin-|Binbrvrt)
            (orgate-Bin)  --  (orgate-Bin-|Binbrvrt)
            (notgate-Bin) -- (notgate-Bin-|Binbrvrt)
            (norgate-Bin) -- (norgate-Bin-|Binbrvrt)

            (nandgate-Bin)
            -- (nandgate-Bin-|Binbrvrt)
            % \constcornerleft {(nandgate-Bin-|Binbrvrt)}
            \constcornerup {(ainit-B)}
            -- (ainit-B)

            % Drawing Pins

            % Ain
            (flad-A-|Ainbrvrt)
            -- ++(-3,0)
            coordinate (Ain)

            % Bin
            (nandgate-Bin-|Binbrvrt)
            coordinate (tmp)
            -- (tmp-|Ain)
            coordinate (Bin)

            % Select lines

            (mux-sl0) ++(0,-2.5)
            coordinate (muxslbr)
            ++(0,-1)
            coordinate (Sl0)

            (mux-sl1)
            \undercornertilldownleft
            {($(mux-sl0)-(mux-sl1)$)} {(muxslbr) ++(-0.5,0)}
            {(muxslbr) ++(-1,0)} {(0,1)}
            coordinate (tmp)
            -- (tmp|-Sl0)
            coordinate (Sl1)

            (mux-sl2)
            \undercornertilldownleft
            {($(mux-sl0)-(mux-sl1)$)} {(muxslbr-|mux-sl2) ++(-0.5,0)}
            {(muxslbr) ++(-2,0)} {(0,1)}
            coordinate (tmp)
            -- (tmp|-Sl0)
            coordinate (Sl2)

            (mux-sl0) -- (Sl0)

            % Clk lines

            (msff-Eclk) ++(0,4)
            coordinate (eclkbr)
            ++ (0,1)
            coordinate (Eclk)

            (msff-Ieclk)
            \undercornertillupright
            {($(msff-Eclk)-(msff-Ieclk)$)} {(eclkbr) ++(0.5,0)}
            {(eclkbr) ++(1,0)} {(0,1)}
            coordinate (tmp)
            -- (tmp|-Eclk)
            coordinate (Ieclk)

            (msff-Eclk)
            -- (Eclk)

            % Reset line

            (ainit-Rst)
            -- (ainit-Rst|-Eclk)
            coordinate (Rstsig)

            % Output lines

            (msffQout)
            -- ++(3.5,0)

            coordinate (Regout)

            (mux-out)
            -- (Regout|-mux-out)
            coordinate (Aluout)

            ;

        \end{circuitikz}
    }
\end{figure}

% \begin{circuitikz}[american]
%
%     \tikzset{barbarrow/.style = {> = {Straight Barb[left, length = 5pt, width = 5pt]}}}
%
%     \draw
%     (0,0) -- ++(1,1)
%     \test {hai} {EC}
%     ;
%
% \end{circuitikz}

\end{document}
