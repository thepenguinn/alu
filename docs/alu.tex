\documentclass[a4paper, 10pt]{article}

\usepackage[siunitx]{circuitikz}
\usepackage[margin = 1in]{geometry}
\usepackage{rotating}
\usepackage{ifthen}

\usetikzlibrary{intersections}

\def\cornerhypo{0.15}
\def\cornerbase{0.4cm}
\def\cV{|-}
\def\cH{-|}

\newcommand\constcorner[2] {
    coordinate (curtmp)
    #2 coordinate (FinalCornerTmp)

    (curtmp#1FinalCornerTmp) coordinate (LastCornerTmp)
    ($(LastCornerTmp)!\cornerbase!0:(FinalCornerTmp)$) coordinate (sectmp)
    ($(LastCornerTmp)!\cornerbase!0:(curtmp)$) coordinate (fsttmp)

    (curtmp) -- (fsttmp) -- (sectmp)
    coordinate (FinalCornerTmp)
}

\newcommand\constcornerhoz [1] {
    \constcorner {\cH} {#1}
}

\newcommand\constcornervrt [1] {
    \constcorner {\cV} {#1}
}

\newcommand\varcornerpos[2] {
    coordinate (curtmp)
    coordinate (FinalCornerTmp) at #2

    coordinate (LastCornerTmp) at (curtmp#1FinalCornerTmp)
    coordinate (cornedgtmp) at ($(FinalCornerTmp)!0.1cm!-45:(LastCornerTmp)$)
    coordinate (fsttmp) at
    (intersection of curtmp--LastCornerTmp and FinalCornerTmp--cornedgtmp)

    (curtmp) -- (fsttmp) -- (FinalCornerTmp)
}

\newcommand\varcornerneg[2] {
    coordinate (curtmp)
    #2 coordinate (FinalCornerTmp)

    (curtmp#1FinalCornerTmp) coordinate (LastCornerTmp)
    ($(FinalCornerTmp)!0.1cm!+45:(LastCornerTmp)$) coordinate (cornedgtmp)
    (intersection of curtmp--LastCornerTmp and FinalCornerTmp--cornedgtmp)
    coordinate (fsttmp)

    (curtmp) -- (fsttmp) -- (FinalCornerTmp)
}

\newcommand\varcornervrtneg[1] {
    \varcornerneg {\cV} {#1}
}

\newcommand\varcornervrtpos[1] {
    \varcornerpos {\cV} {#1}
}

\newcommand\varcornerhozneg[1] {
    \varcornerneg {\cH} {#1}
}

\newcommand\varcornerhozpos[1] {
    \varcornerpos {\cH} {#1}
}

% \ctikzsubcircuitdef{msFlop} {EC, IEC, D, RST, Q} {
% \ctikzsubcircuitdef{test} {EC} {
%     coordinate(#1-EC)
%     \corneranticlk {++(0,4)}
%     \corneranticlk {++(4,0)}
%     -- ++(4,0)
% }

\tikzset{fullAdder/.style = {muxdemux, muxdemux def = {
    NL = 3, NR = 3, NB = 0, NT = 0,
    Lh = 3, Rh = 3,
    w = 5 },
draw only right pins = {1, 3}
}
}

\tikzset{aluInit/.style = {muxdemux, muxdemux def = {
    NL = 3, NR = 2, NB = 6, NT = 2,
    Lh = 3, Rh = 3,
    % Lh = 5.35, Rh = 3,
    w = 5
},
draw only top pins = {1},
draw only bottom pins = {5, 6},
}
}


\tikzset{logicOp/.style = {muxdemux, muxdemux def = {
    NL = 3, NR = 1, NB = 0, NT = 0,
    Lh = 3, Rh = 3,
    w = 5
    },
    }
}

\def\fulladdername{\texttt{ADDER}}
% \def\logicopstyle{\texttt{}}


% \ctikzsubcircuitdef{onebitalu} {ina, inb, eclk, ieclk, rst, op0, op1, op2, regout, aluout} {
\ctikzsubcircuitdef{onebitalu} {L} {
    coordinate (#1-L)
    node [fullAdder] (#1-flad) {\fulladdername}
}


\ctikzsubcircuitactivate{test}
\ctikzsubcircuitactivate{onebitalu}


\begin{document}

\section{1 bit ALU}

%  % \ctikzset{muxdemuxes/thickness = 3, multipoles/external pins thickness = 1.5, bipoles/tline/width = 3}

%  \draw[semithick]

%  node [muxdemux, rotate = -90, muxdemux def = {NL=10, NR=3, NT=0, NB=4,
%  w = 2.5, Lh = 10, Rh = 6.5, inset Lh = 3.5, inset w = .75, inset Rh = 1.75,
%  square pins=1
%  },
%  draw only left pins={2-9},
%  draw only right pins={1-2},
%  draw only bottom pins={2-4}]
%  (mux) {
%      \begin{turn}{90}
%          \texttt{MUX8}
%      \end{turn}
%  }

%  % (mux.rpin 3) -- ++(1,1)
%  ;

% \tikzset{blockdef-above/.style={
%     {Straight Barb[harpoon, right, length=-0.2cm]}-{Straight Barb[harpoon, left, length=-0.2cm]},
%     black,
%     }}

% \tikzset{barbarrow/.style = {> = {Straight Barb[right, length = 5pt, width = 5pt]}}}

\begin{circuitikz}[american]

    \draw
    (0,0)
    node (flad) [fullAdder, anchor = lpin 1]{\texttt{\fulladdername}}
    (flad.rpin 1) -- ++(0.5,0)
    node (ainit) [aluInit, anchor = lpin 3]{\texttt{ALUINIT}}

    ;


\end{circuitikz}

% \begin{circuitikz}[american]
%
%     \tikzset{barbarrow/.style = {> = {Straight Barb[left, length = 5pt, width = 5pt]}}}
%
%     \draw
%     (0,0) -- ++(1,1)
%     \test {hai} {EC}
%     ;
%
% \end{circuitikz}

\end{document}
