\tikzset{haltUnit/.style = {muxdemux, muxdemux def = {
    NL = 0, NR = 2, NB = 6, NT = 6,
    Lh = 3, Rh = 3,
    w = 5 },
    draw only right pins = {1},
    draw only top pins = {1,2,3},
    draw only bottom pins = {3,4}, }
}

\ctikzsubcircuitdef{sctkhaltunit} {
    notreclk, notdfeclk, notdreclk, rgfsrq, muxlast, reclk,
    bnotreclk, bnotdfeclk, bnotdreclk, brgfsrq, bmuxlast, breclk%
} {
    node (#1) [haltUnit]{}%
    \markcoordinate {#1} {notreclk}  {tpin 1}
    \markcoordinate {#1} {notdfeclk} {tpin 2}
    \markcoordinate {#1} {notdreclk} {tpin 3}
    \markcoordinate {#1} {rgfsrq}    {bpin 3}
    \markcoordinate {#1} {muxlast}   {bpin 4}
    \markcoordinate {#1} {reclk}     {rpin 1}
    \markcoordinate {#1} {fsrq}      {rpin 2}
    (#1-breclk) node [circleinv, fill = circleinvfillcolor, anchor = left] {}
    (#1-bmuxlast) node [circleinv, fill = circleinvfillcolor, anchor = top] {}
}

\ctikzsubcircuitactivate{sctkhaltunit}

\newcommand\sctklabelhaltUnit[2] {
    (#1.center) node[]{#2}
}
