\tikzset{muxFirstLayer/.style = {muxdemux, muxdemux def = {
    NL = 6, NR = 6, NB = 2, NT = 2,
    Lh = 5, Rh = 5,
    % Lh = 5.35, Rh = 3,
    w = 3 },
    draw only left pins = {2-5},
    draw only right pins = {}
    }
}

\ctikzsubcircuitdef{sctkmuxfirstlayer} {
    in0, in1, out0, out1,
    lsl0, lsl1, lsl2, lsl3,
    rsl0, rsl1, rsl2, rsl3,
    blsl0, blsl1, blsl2, blsl3,
    bin0, bin1, bout0, bout1%
} {
    node (#1) [muxFirstLayer]{}%
    \markcoordinate {#1} {lsl0}  {lpin 5}
    \markcoordinate {#1} {lsl1}  {lpin 4}
    \markcoordinate {#1} {lsl2}  {lpin 3}
    \markcoordinate {#1} {lsl3}  {lpin 2}
    %%%
    (#1.rpin 5) coordinate (#1-rsl0)
    (#1.rpin 4) coordinate (#1-rsl1)
    (#1.rpin 3) coordinate (#1-rsl2)
    (#1.rpin 2) coordinate (#1-rsl3)
    %%%
    \markcoordinate {#1} {in0}   {tpin 1}
    \markcoordinate {#1} {in1}   {tpin 2}
    \markcoordinate {#1} {out0}  {bpin 1}
    \markcoordinate {#1} {out1}  {bpin 2}
}

\ctikzsubcircuitactivate{sctkmuxfirstlayer}

\newcommand\sctklabelmuxfirstlayer[2] {
    (#1.center) node[]{#2}
}
