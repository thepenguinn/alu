\documentclass[../main]{subfiles}

\ifSubfilesClassLoaded{
    \tikzset{muxFirstLayer/.style = {muxdemux, muxdemux def = {
    NL = 6, NR = 6, NB = 2, NT = 2,
    Lh = 5, Rh = 5,
    % Lh = 5.35, Rh = 3,
    w = 3 },
    draw only left pins = {2-5},
    draw only right pins = {}
    }
}

\tikzset{muxSecondLayer/.style = {muxdemux, muxdemux def = {
    NL = 4, NR = 4, NB = 3, NT = 8,
    Lh = 4, Rh = 4,
    % Lh = 5.35, Rh = 3,
    w = 8 },
    draw only bottom pins = {2}
    }
}

    \ctikzsubcircuitdef{subctkfullAdder} {
    a, a, cin, cout, sum,
    ba, bb, bcin, bcout, bsum%
} {
    node (#1) [fullAdder, anchor = lpin 1]{}%
    \markcoordinate {#1} {b}    {lpin 1}
    \markcoordinate {#1} {a}    {lpin 3}
    \markcoordinate {#1} {cin}  {lpin 2}
    \markcoordinate {#1} {sum}  {rpin 3}
    \markcoordinate {#1} {cout} {rpin 1}%
}

\ctikzsubcircuitdef{subctkaluInit} {
    b, cin, op, rst, bout, cout,
    bb, bcin, bop, brst, bbout, bcout%
} {
    node (#1) [aluInit, anchor = lpin 1]{}%
    \markcoordinate {#1} {b}    {lpin 1}
    \markcoordinate {#1} {cin}  {lpin 2}
    \markcoordinate {#1} {bout} {rpin 1}
    \markcoordinate {#1} {cout} {rpin 2}
    \markcoordinate {#1} {op}   {bpin 5}
    \markcoordinate {#1} {rst}  {tpin 1}%
    (#1-bcout) node [circleinv, anchor = left] {}
}

\ctikzsubcircuitdef{subctklogicOp} {
    ain, bin, out,
    bain, bbin, bout%
} {
    node (#1) [logicOp, anchor = lpin 1]{}%
    \markcoordinate {#1} {ain} {lpin 2}
    \markcoordinate {#1} {bin} {lpin 3}
    \markcoordinate {#1} {out} {rpin 1}%
}

\ctikzsubcircuitdef{subctkmuxEight} {
    in0, in2, in3, in4, in5, in6, in7, sl0, sl1, sl2, out, and01,
    bin0, bin2, bin3, bin4, bin5, bin6, bin7, bsl0, bsl1, bsl2, bout, band01,
    pin0, pin2, pin3, pin4, pin5, pin6, pin7, psl0, psl1, psl2, pout, pand01%
} {
    node (#1) [mux08, anchor = lpin 1] {}%
    \markcoordinate {#1}   {in0} {lpin 2}
    \markcoordinate {#1}   {in2} {lpin 4}
    \markcoordinate {#1}   {in3} {lpin 5}
    \markcoordinate {#1}   {in4} {lpin 6}
    \markcoordinate {#1}   {in5} {lpin 7}
    \markcoordinate {#1}   {in6} {lpin 8}
    \markcoordinate {#1}   {in7} {lpin 9}%
    \markcoordinate {#1}   {sl0} {bpin 4}
    \markcoordinate {#1}   {sl1} {bpin 3}
    \markcoordinate {#1}   {sl2} {bpin 2}%
    \markcoordinate {#1} {and01} {rpin 1}
    \markcoordinate {#1}   {out} {rpin 2}
    (#1-bout) node [circleinv, anchor = left] {}
    (#1-band01) node [circleinv, anchor = left] {}
}

\ctikzsubcircuitactivate{subctkfullAdder}
\ctikzsubcircuitactivate{subctkaluInit}
\ctikzsubcircuitactivate{subctklogicOp}
\ctikzsubcircuitactivate{subctkmuxEight}

\newcommand\fullAdder[3] {
    \subctkfullAdder{#1} {#2}
    (#1.center) node[]{#3}
}

\newcommand\aluInit[3] {
    \subctkaluInit{#1} {#2}
    (#1.center) node[]{#3}
}

\newcommand\logicOp[3] {
    \subctklogicOp{#1} {#2}
    (#1.center) node[]{#3}
}

\newcommand\muxEight[3] {
    \subctkmuxEight{#1} {#2}
    (#1.narrow center) node[]{#3}
}

\ctikzsubcircuitdef{subctkaluOneBit} {
    north, south, east, west,
    northeast, northwest, southeast, southwest,
    center,
    ain, bin, reclk, feclk, op0, op1, op2, rst, regout, aluout%
} {
    \fullAdder{#1-flad}{cin} {}
    (#1-flad-cout)
    \constcornerhoz {++(0.75,1)} -- ++(0.25,0)
    %%
    \aluInit{#1-ainit}{cin} {}
    (#1-ainit-cout)
    \constcornerhoz {++(0.75,-1)} -- ++(0.25,0)
    %%
    \msFlipFlop{#1-msff}{d} {}
    %%
    (#1-ainit-bout) ++(0,2)
    coordinate (#1-ainitboutheight)
    %%
    (#1-ainit-bout)
    \constcornerright {++(0.75,1)}
    %%
    \constcornerup {(#1-ainitboutheight)}
    \constcornerfromleft {(#1-flad-pb)}
    coordinate (#1-fladleftvrt)
    \constcornerdown {(#1-flad-b)}
    -- (#1-flad-b)
    %%
    (#1-msff-q) -- (#1-msff-pq) -- ++(0.5,0)
    coordinate (#1-msffqout)
    -- (#1-ainitboutheight-|#1-msffqout)
    \constcornerup {++($(#1-flad-b)-(#1-flad-cin)$) ++(-1,0)}
    %%
    \aroundcornerleftdown {($(#1-flad-b)-(#1-flad-cin)$)} {(#1-fladleftvrt)}
    \aroundcornertilldownright {($(#1-flad-b)-(#1-flad-cin)$)} {(#1-flad-pb)}
    {(#1-flad-cin)} {(1,0)}
    -- (#1-flad-cin)
    %%
    (#1-msff-rst)
    \constcornerdown {++(1,-0.75)}
    \constcornerright {++(2,-1)}
    coordinate (#1-msffrstvrt)
    \constcornerdown {++(-1,-3)}
    -- ++(-0.5,0)
    %%
    coordinate (#1-tmp)
    \muxEight {#1-mux} {pand01} {}
    (#1-tmp)
    -- (#1-mux-and01)
    %%
    ($(#1-mux-in5)!0.5!(#1-mux-in6)$)
    coordinate (#1-muxbotmid)
    (#1-muxbotmid-|#1-ainit-op) ++(-0.75,0)
    coordinate (#1-muxbotcnt)
    %%
    (#1-muxbotcnt) ++(0,0.75)
    \logicOp {#1-orgate} {out} {}
    %%
    (#1-muxbotcnt) ++(0,-2)
    \logicOp {#1-nandgate} {out} {}
    %%
    (#1-muxbotcnt-|#1-orgate-ain) ++(0,2)
    \logicOp {#1-notgate} {out} {}
    %%
    (#1-muxbotcnt-|#1-orgate-ain) ++(0,-0.75)
    \logicOp {#1-norgate} {out} {}
    %%
    (#1-muxbotcnt) ++(1.5,0)
    coordinate (#1-muxbotmid)
    %%
    (#1-orgate-out)
    \varcornerrightdown {(#1-muxbotmid|-#1-mux-in5)}
    -- (#1-mux-in5)
    %%
    (#1-norgate-out)
    \varcornerrightup {(#1-muxbotmid|-#1-mux-in6)}
    -- (#1-mux-in6)
    %%
    (#1-notgate-out)
    \aroundcornertillrightdown
    {($(#1-mux-in5)-(#1-mux-in6)$)} {(#1-muxbotmid|-#1-mux-in5)}
    {(#1-mux-in4)} {(1,0)}
    -- (#1-mux-in4)
    %%
    (#1-nandgate-out)
    \aroundcornertillrightup
    {($(#1-mux-in5)-(#1-mux-in6)$)} {(#1-muxbotmid|-#1-mux-in6)}
    {(#1-mux-in7)} {(1,0)}
    -- (#1-mux-in7)
    %%
    (#1-orgate-out) ++(0,3)
    \logicOp {#1-andgate} {out} {}
    %%
    (#1-notgate-out) ++(0,3)
    \logicOp {#1-xorgate} {out} {}
    %%
    (#1-andgate-out)
    \varcornerrightdown {(#1-muxbotmid|-#1-mux-in3)}
    -- (#1-mux-in3)
    %%
    (#1-xorgate-out)
    \aroundcornertillrightdown
    {($(#1-mux-in5)-(#1-mux-in6)$)} {(#1-muxbotmid|-#1-mux-in3)}
    {(#1-mux-in2)} {(1,0)}
    -- (#1-mux-in2)
    %%
    (#1-flad-sum)
    \aroundcornertillrightdown
    {($(#1-mux-in0)-(#1-mux-in3)$)} {(#1-muxbotmid|-#1-mux-in3)}
    {(#1-mux-in0)} {(1,0)}
    -- (#1-mux-in0)
    %%
    %% ainit op
    %%
    (#1-ainit-op)
    \varcornerdownright {(#1-msff-d|-#1-msffrstvrt) ++(0,-1)}
    \constcornerright {++(6,-1)}
    \constcornerfromdown {(#1-mux-sl0)}
    coordinate (#1-tmp)
    -- (#1-tmp-|#1-mux-sl0)
    %%
    % a input
    %%
    (#1-xorgate-ain)
    -- ++(-2,0)
    coordinate (#1-ainbrvrt)
    %%
    (#1-andgate-ain) -- (#1-andgate-ain-|#1-ainbrvrt)
    (#1-orgate-ain)  --  (#1-orgate-ain-|#1-ainbrvrt)
    (#1-notgate-ain) -- (#1-notgate-ain-|#1-ainbrvrt)
    (#1-norgate-ain) -- (#1-norgate-ain-|#1-ainbrvrt)
    %%
    (#1-nandgate-ain)
    \constcornerleft {(#1-nandgate-ain-|#1-ainbrvrt)}
    -- (#1-flad-a-|#1-ainbrvrt)
    -- (#1-flad-a)
    %%
    % b input
    %%
    (#1-xorgate-bin)
    -- ++(-2.75,0)
    coordinate (#1-binbrvrt)
    %%
    (#1-andgate-bin) -- (#1-andgate-bin-|#1-binbrvrt)
    (#1-orgate-bin)  --  (#1-orgate-bin-|#1-binbrvrt)
    (#1-notgate-bin) -- (#1-notgate-bin-|#1-binbrvrt)
    (#1-norgate-bin) -- (#1-norgate-bin-|#1-binbrvrt)
    %%
    (#1-nandgate-bin)
    -- (#1-nandgate-bin-|#1-binbrvrt)
    % \constcornerleft {(nandgate-bin-|binbrvrt)}
    \constcornerup {(#1-ainit-b)}
    -- (#1-ainit-b)
    %%
    % drawing pins
    %%
    % ain
    (#1-flad-a-|#1-ainbrvrt)
    -- ++(-2,0)
    coordinate (#1-ain)
    %%
    % bin
    (#1-nandgate-bin-|#1-binbrvrt)
    coordinate (#1-tmp)
    -- (#1-tmp-|#1-ain)
    coordinate (#1-bin)
    %%
    % select lines
    %%
    (#1-mux-sl0) ++(0,-2.5)
    coordinate (#1-muxslbr)
    ++(0,-1)
    coordinate (#1-op0)
    %%
    (#1-mux-sl1)
    \undercornertilldownleft
    {($(#1-mux-sl0)-(#1-mux-sl1)$)} {(#1-muxslbr) ++(-0.5,0)}
    {(#1-muxslbr) ++(-1,0)} {(0,1)}
    coordinate (#1-tmp)
    -- (#1-tmp|-#1-op0)
    coordinate (#1-op1)
    %%
    (#1-mux-sl2)
    \undercornertilldownleft
    {($(#1-mux-sl0)-(#1-mux-sl1)$)} {(#1-muxslbr-|#1-mux-sl2) ++(-0.5,0)}
    {(#1-muxslbr) ++(-2,0)} {(0,1)}
    coordinate (#1-tmp)
    -- (#1-tmp|-#1-op0)
    coordinate (#1-op2)
    %%
    (#1-mux-sl0) -- (#1-op0)
    %%
    % clk lines
    %%
    (#1-msff-reclk) ++(0,4)
    coordinate (#1-reclkbr)
    ++ (0,1)
    coordinate (#1-reclk)
    %%
    (#1-msff-feclk)
    \undercornertillupright
    {($(#1-msff-reclk)-(#1-msff-feclk)$)} {(#1-reclkbr) ++(0.5,0)}
    {(#1-reclkbr) ++(1,0)} {(0,1)}
    coordinate (#1-tmp)
    -- (#1-tmp|-#1-reclk)
    coordinate (#1-feclk)
    %%
    (#1-msff-reclk)
    -- (#1-reclk)
    %%
    % reset line
    %%
    (#1-ainit-rst)
    -- (#1-ainit-rst|-#1-reclk)
    coordinate (#1-rst)
    %%
    % output lines
    %%
    (#1-msffqout)
    -- ++(3.5,0)
    %%
    coordinate (#1-regout)
    %%
    (#1-mux-out)
    -- (#1-regout|-#1-mux-out)
    coordinate (#1-aluout)
    %% Marking geo coords
    \markgeocoordinate {#1}
    {(#1-rst)} {(#1-op0)}
    {(#1-ain)} {(#1-aluout)}
}

\ctikzsubcircuitactivate{subctkaluOneBit}

% Name Anchor
\newcommand\aluOneBit[2] {
    \subctkaluOneBit {#1} {#2}
}

% Name Adder Aluinit Msff Mux
\newcommand\labelaluOneBit[5] {

    (#1-flad.center)       node[]{#2}
    (#1-ainit.center)      node[]{#3}
    (#1-msff.center)       node[]{#4}
    (#1-mux.narrow center) node[]{#5}
}

% xor and not or nor nand
\newcommand\labelaluOneBitLogicOp[7] {
    (#1-xorgate.center)  node[]{#2}
    (#1-andgate.center)  node[]{#3}
    (#1-notgate.center)  node[]{#4}
    (#1-orgate.center)   node[]{#5}
    (#1-norgate.center)  node[]{#6}
    (#1-nandgate.center) node[]{#7}
}

    %% um, why did I add this...
    %% \tikzset{muxFirstLayer/.style = {muxdemux, muxdemux def = {
    NL = 6, NR = 6, NB = 2, NT = 2,
    Lh = 5, Rh = 5,
    % Lh = 5.35, Rh = 3,
    w = 3 },
    draw only left pins = {2-5},
    draw only right pins = {}
    }
}

\tikzset{muxSecondLayer/.style = {muxdemux, muxdemux def = {
    NL = 4, NR = 4, NB = 3, NT = 8,
    Lh = 4, Rh = 4,
    % Lh = 5.35, Rh = 3,
    w = 8 },
    draw only bottom pins = {2}
    }
}

    %% \ctikzsubcircuitdef{subctkfullAdder} {
    a, a, cin, cout, sum,
    ba, bb, bcin, bcout, bsum%
} {
    node (#1) [fullAdder, anchor = lpin 1]{}%
    \markcoordinate {#1} {b}    {lpin 1}
    \markcoordinate {#1} {a}    {lpin 3}
    \markcoordinate {#1} {cin}  {lpin 2}
    \markcoordinate {#1} {sum}  {rpin 3}
    \markcoordinate {#1} {cout} {rpin 1}%
}

\ctikzsubcircuitdef{subctkaluInit} {
    b, cin, op, rst, bout, cout,
    bb, bcin, bop, brst, bbout, bcout%
} {
    node (#1) [aluInit, anchor = lpin 1]{}%
    \markcoordinate {#1} {b}    {lpin 1}
    \markcoordinate {#1} {cin}  {lpin 2}
    \markcoordinate {#1} {bout} {rpin 1}
    \markcoordinate {#1} {cout} {rpin 2}
    \markcoordinate {#1} {op}   {bpin 5}
    \markcoordinate {#1} {rst}  {tpin 1}%
    (#1-bcout) node [circleinv, anchor = left] {}
}

\ctikzsubcircuitdef{subctklogicOp} {
    ain, bin, out,
    bain, bbin, bout%
} {
    node (#1) [logicOp, anchor = lpin 1]{}%
    \markcoordinate {#1} {ain} {lpin 2}
    \markcoordinate {#1} {bin} {lpin 3}
    \markcoordinate {#1} {out} {rpin 1}%
}

\ctikzsubcircuitdef{subctkmuxEight} {
    in0, in2, in3, in4, in5, in6, in7, sl0, sl1, sl2, out, and01,
    bin0, bin2, bin3, bin4, bin5, bin6, bin7, bsl0, bsl1, bsl2, bout, band01,
    pin0, pin2, pin3, pin4, pin5, pin6, pin7, psl0, psl1, psl2, pout, pand01%
} {
    node (#1) [mux08, anchor = lpin 1] {}%
    \markcoordinate {#1}   {in0} {lpin 2}
    \markcoordinate {#1}   {in2} {lpin 4}
    \markcoordinate {#1}   {in3} {lpin 5}
    \markcoordinate {#1}   {in4} {lpin 6}
    \markcoordinate {#1}   {in5} {lpin 7}
    \markcoordinate {#1}   {in6} {lpin 8}
    \markcoordinate {#1}   {in7} {lpin 9}%
    \markcoordinate {#1}   {sl0} {bpin 4}
    \markcoordinate {#1}   {sl1} {bpin 3}
    \markcoordinate {#1}   {sl2} {bpin 2}%
    \markcoordinate {#1} {and01} {rpin 1}
    \markcoordinate {#1}   {out} {rpin 2}
    (#1-bout) node [circleinv, anchor = left] {}
    (#1-band01) node [circleinv, anchor = left] {}
}

\ctikzsubcircuitactivate{subctkfullAdder}
\ctikzsubcircuitactivate{subctkaluInit}
\ctikzsubcircuitactivate{subctklogicOp}
\ctikzsubcircuitactivate{subctkmuxEight}

\newcommand\fullAdder[3] {
    \subctkfullAdder{#1} {#2}
    (#1.center) node[]{#3}
}

\newcommand\aluInit[3] {
    \subctkaluInit{#1} {#2}
    (#1.center) node[]{#3}
}

\newcommand\logicOp[3] {
    \subctklogicOp{#1} {#2}
    (#1.center) node[]{#3}
}

\newcommand\muxEight[3] {
    \subctkmuxEight{#1} {#2}
    (#1.narrow center) node[]{#3}
}

\ctikzsubcircuitdef{subctkaluOneBit} {
    north, south, east, west,
    northeast, northwest, southeast, southwest,
    center,
    ain, bin, reclk, feclk, op0, op1, op2, rst, regout, aluout%
} {
    \fullAdder{#1-flad}{cin} {}
    (#1-flad-cout)
    \constcornerhoz {++(0.75,1)} -- ++(0.25,0)
    %%
    \aluInit{#1-ainit}{cin} {}
    (#1-ainit-cout)
    \constcornerhoz {++(0.75,-1)} -- ++(0.25,0)
    %%
    \msFlipFlop{#1-msff}{d} {}
    %%
    (#1-ainit-bout) ++(0,2)
    coordinate (#1-ainitboutheight)
    %%
    (#1-ainit-bout)
    \constcornerright {++(0.75,1)}
    %%
    \constcornerup {(#1-ainitboutheight)}
    \constcornerfromleft {(#1-flad-pb)}
    coordinate (#1-fladleftvrt)
    \constcornerdown {(#1-flad-b)}
    -- (#1-flad-b)
    %%
    (#1-msff-q) -- (#1-msff-pq) -- ++(0.5,0)
    coordinate (#1-msffqout)
    -- (#1-ainitboutheight-|#1-msffqout)
    \constcornerup {++($(#1-flad-b)-(#1-flad-cin)$) ++(-1,0)}
    %%
    \aroundcornerleftdown {($(#1-flad-b)-(#1-flad-cin)$)} {(#1-fladleftvrt)}
    \aroundcornertilldownright {($(#1-flad-b)-(#1-flad-cin)$)} {(#1-flad-pb)}
    {(#1-flad-cin)} {(1,0)}
    -- (#1-flad-cin)
    %%
    (#1-msff-rst)
    \constcornerdown {++(1,-0.75)}
    \constcornerright {++(2,-1)}
    coordinate (#1-msffrstvrt)
    \constcornerdown {++(-1,-3)}
    -- ++(-0.5,0)
    %%
    coordinate (#1-tmp)
    \muxEight {#1-mux} {pand01} {}
    (#1-tmp)
    -- (#1-mux-and01)
    %%
    ($(#1-mux-in5)!0.5!(#1-mux-in6)$)
    coordinate (#1-muxbotmid)
    (#1-muxbotmid-|#1-ainit-op) ++(-0.75,0)
    coordinate (#1-muxbotcnt)
    %%
    (#1-muxbotcnt) ++(0,0.75)
    \logicOp {#1-orgate} {out} {}
    %%
    (#1-muxbotcnt) ++(0,-2)
    \logicOp {#1-nandgate} {out} {}
    %%
    (#1-muxbotcnt-|#1-orgate-ain) ++(0,2)
    \logicOp {#1-notgate} {out} {}
    %%
    (#1-muxbotcnt-|#1-orgate-ain) ++(0,-0.75)
    \logicOp {#1-norgate} {out} {}
    %%
    (#1-muxbotcnt) ++(1.5,0)
    coordinate (#1-muxbotmid)
    %%
    (#1-orgate-out)
    \varcornerrightdown {(#1-muxbotmid|-#1-mux-in5)}
    -- (#1-mux-in5)
    %%
    (#1-norgate-out)
    \varcornerrightup {(#1-muxbotmid|-#1-mux-in6)}
    -- (#1-mux-in6)
    %%
    (#1-notgate-out)
    \aroundcornertillrightdown
    {($(#1-mux-in5)-(#1-mux-in6)$)} {(#1-muxbotmid|-#1-mux-in5)}
    {(#1-mux-in4)} {(1,0)}
    -- (#1-mux-in4)
    %%
    (#1-nandgate-out)
    \aroundcornertillrightup
    {($(#1-mux-in5)-(#1-mux-in6)$)} {(#1-muxbotmid|-#1-mux-in6)}
    {(#1-mux-in7)} {(1,0)}
    -- (#1-mux-in7)
    %%
    (#1-orgate-out) ++(0,3)
    \logicOp {#1-andgate} {out} {}
    %%
    (#1-notgate-out) ++(0,3)
    \logicOp {#1-xorgate} {out} {}
    %%
    (#1-andgate-out)
    \varcornerrightdown {(#1-muxbotmid|-#1-mux-in3)}
    -- (#1-mux-in3)
    %%
    (#1-xorgate-out)
    \aroundcornertillrightdown
    {($(#1-mux-in5)-(#1-mux-in6)$)} {(#1-muxbotmid|-#1-mux-in3)}
    {(#1-mux-in2)} {(1,0)}
    -- (#1-mux-in2)
    %%
    (#1-flad-sum)
    \aroundcornertillrightdown
    {($(#1-mux-in0)-(#1-mux-in3)$)} {(#1-muxbotmid|-#1-mux-in3)}
    {(#1-mux-in0)} {(1,0)}
    -- (#1-mux-in0)
    %%
    %% ainit op
    %%
    (#1-ainit-op)
    \varcornerdownright {(#1-msff-d|-#1-msffrstvrt) ++(0,-1)}
    \constcornerright {++(6,-1)}
    \constcornerfromdown {(#1-mux-sl0)}
    coordinate (#1-tmp)
    -- (#1-tmp-|#1-mux-sl0)
    %%
    % a input
    %%
    (#1-xorgate-ain)
    -- ++(-2,0)
    coordinate (#1-ainbrvrt)
    %%
    (#1-andgate-ain) -- (#1-andgate-ain-|#1-ainbrvrt)
    (#1-orgate-ain)  --  (#1-orgate-ain-|#1-ainbrvrt)
    (#1-notgate-ain) -- (#1-notgate-ain-|#1-ainbrvrt)
    (#1-norgate-ain) -- (#1-norgate-ain-|#1-ainbrvrt)
    %%
    (#1-nandgate-ain)
    \constcornerleft {(#1-nandgate-ain-|#1-ainbrvrt)}
    -- (#1-flad-a-|#1-ainbrvrt)
    -- (#1-flad-a)
    %%
    % b input
    %%
    (#1-xorgate-bin)
    -- ++(-2.75,0)
    coordinate (#1-binbrvrt)
    %%
    (#1-andgate-bin) -- (#1-andgate-bin-|#1-binbrvrt)
    (#1-orgate-bin)  --  (#1-orgate-bin-|#1-binbrvrt)
    (#1-notgate-bin) -- (#1-notgate-bin-|#1-binbrvrt)
    (#1-norgate-bin) -- (#1-norgate-bin-|#1-binbrvrt)
    %%
    (#1-nandgate-bin)
    -- (#1-nandgate-bin-|#1-binbrvrt)
    % \constcornerleft {(nandgate-bin-|binbrvrt)}
    \constcornerup {(#1-ainit-b)}
    -- (#1-ainit-b)
    %%
    % drawing pins
    %%
    % ain
    (#1-flad-a-|#1-ainbrvrt)
    -- ++(-2,0)
    coordinate (#1-ain)
    %%
    % bin
    (#1-nandgate-bin-|#1-binbrvrt)
    coordinate (#1-tmp)
    -- (#1-tmp-|#1-ain)
    coordinate (#1-bin)
    %%
    % select lines
    %%
    (#1-mux-sl0) ++(0,-2.5)
    coordinate (#1-muxslbr)
    ++(0,-1)
    coordinate (#1-op0)
    %%
    (#1-mux-sl1)
    \undercornertilldownleft
    {($(#1-mux-sl0)-(#1-mux-sl1)$)} {(#1-muxslbr) ++(-0.5,0)}
    {(#1-muxslbr) ++(-1,0)} {(0,1)}
    coordinate (#1-tmp)
    -- (#1-tmp|-#1-op0)
    coordinate (#1-op1)
    %%
    (#1-mux-sl2)
    \undercornertilldownleft
    {($(#1-mux-sl0)-(#1-mux-sl1)$)} {(#1-muxslbr-|#1-mux-sl2) ++(-0.5,0)}
    {(#1-muxslbr) ++(-2,0)} {(0,1)}
    coordinate (#1-tmp)
    -- (#1-tmp|-#1-op0)
    coordinate (#1-op2)
    %%
    (#1-mux-sl0) -- (#1-op0)
    %%
    % clk lines
    %%
    (#1-msff-reclk) ++(0,4)
    coordinate (#1-reclkbr)
    ++ (0,1)
    coordinate (#1-reclk)
    %%
    (#1-msff-feclk)
    \undercornertillupright
    {($(#1-msff-reclk)-(#1-msff-feclk)$)} {(#1-reclkbr) ++(0.5,0)}
    {(#1-reclkbr) ++(1,0)} {(0,1)}
    coordinate (#1-tmp)
    -- (#1-tmp|-#1-reclk)
    coordinate (#1-feclk)
    %%
    (#1-msff-reclk)
    -- (#1-reclk)
    %%
    % reset line
    %%
    (#1-ainit-rst)
    -- (#1-ainit-rst|-#1-reclk)
    coordinate (#1-rst)
    %%
    % output lines
    %%
    (#1-msffqout)
    -- ++(3.5,0)
    %%
    coordinate (#1-regout)
    %%
    (#1-mux-out)
    -- (#1-regout|-#1-mux-out)
    coordinate (#1-aluout)
    %% Marking geo coords
    \markgeocoordinate {#1}
    {(#1-rst)} {(#1-op0)}
    {(#1-ain)} {(#1-aluout)}
}

\ctikzsubcircuitactivate{subctkaluOneBit}

% Name Anchor
\newcommand\aluOneBit[2] {
    \subctkaluOneBit {#1} {#2}
}

% Name Adder Aluinit Msff Mux
\newcommand\labelaluOneBit[5] {

    (#1-flad.center)       node[]{#2}
    (#1-ainit.center)      node[]{#3}
    (#1-msff.center)       node[]{#4}
    (#1-mux.narrow center) node[]{#5}
}

% xor and not or nor nand
\newcommand\labelaluOneBitLogicOp[7] {
    (#1-xorgate.center)  node[]{#2}
    (#1-andgate.center)  node[]{#3}
    (#1-notgate.center)  node[]{#4}
    (#1-orgate.center)   node[]{#5}
    (#1-norgate.center)  node[]{#6}
    (#1-nandgate.center) node[]{#7}
}

    %% \tikzexternalize[prefix=figcache/]
    \usepackage{catppuccintheme}
    \usemintedstyle{catppuccin-mocha}
    \ctikzset{open poles fill = black}
} {
}

\begin{document}

\section{Basic Circuits}

\subsection {FlipFlops}

\subsubsection {SR Latch}

\begin{minted} {verilog}
    module sr_latch(input logic set, reset,
        output logic q, qbar);

        nor #3 (q, qbar, reset);
        nor #3 (qbar, q, set);

    endmodule
\end{minted}

\subsubsection {SR FlipFlop}

\begin{minted} {verilog}
    module sr_flipflop(input logic set, reset, clk,
        output logic q, qbar);

        /*
         * With active low clk
         * */

        wire srlset, srlreset;

        sr_latch srl (
            .set(srlset),
            .reset(srlreset),
            .q(q),
            .qbar(qbar)
        );

        nor #3 (srlreset, set, clk);
        nor #3 (srlset, reset, clk);

    endmodule
\end{minted}

\subsubsection {D FlipFlop}

\begin{minted} {verilog}
    module d_flipflop(input logic data, clk,
        output logic q, qbar);

        /*
         * With active low clk
         * */

         wire notdata;

        sr_flipflop srff (
            .set(data),
            .reset(notdata),
            .clk(clk),
            .q(q),
            .qbar(qbar)
        );

        nor #3 (notdata, data, data);

    endmodule
\end{minted}

\subsubsection {D FlipFlop with Synchronous Reset}

\begin{minted} {verilog}
    module d_flipflop_sr(input logic data, rst, clk,
        output logic q, qbar);

        /*
         * With active low clk and active low data
         * */

        wire acdata;

        d_flipflop dl (
            .data(acdata),
            .clk(clk),
            .q(q),
            .qbar(qbar)
        );

        nor #3 (acdata, data, rst);

    endmodule
\end{minted}

\subsubsection {MS FlipFlop}

\begin{minted} {verilog}
    module ms_flipflop(input logic data, rst, reclk, feclk,
        output logic q, qbar);

        /*
         * With synchronous reset, active low data,
         * active low reclk, and feclk
         * */

        wire sdata, notsdata;

        d_flipflop_sr master (
            .data(data),
            .clk(reclk),
            .q(sdata),
            .qbar(notsdata),
            .rst(rst)
        );

        sr_flipflop slave (
            .set(sdata),
            .reset(notsdata),
            .clk(feclk),
            .q(q),
            .qbar(qbar)
        );

    endmodule
\end{minted}

\subsection{Multiplexer}

\subsubsection {Five Input NOR Gate from Two Input NOR Gate}

\begin{minted} {verilog}
    module nor5input(input logic in0, in1, in2, in3, in4,
        output logic out, f4high);

        wire nor01, nor23;
        wire or01, or23;
        wire nor0123;
        wire or0123;

        nor #3 (nor01, in0, in1);
        nor #3 (or01, nor01, nor01);

        nor #3 (nor23, in2, in3);
        nor #3 (or23, nor23, nor23);

        nor #3 (nor0123, or01, or23);
        nor #3 (or0123, nor0123, nor0123);

        nor #3 (out, or0123, in4);

        assign f4high = or0123;

    endmodule
\end{minted}

\subsubsection {Four Input NOR Gate from Two Input NOR Gate}

\begin{minted} [mathescape] {verilog}
    module nor4input(input logic in0, in1, in2, in3,
        output logic out);

        wire nor01, nor23;
        wire or01, or23;

        nor #3 (nor01, in0, in1);
        nor #3 (or01, nor01, nor01);

        nor #3 (nor23, in2, in3);
        nor #3 (or23, nor23, nor23);

        nor #3 (out, or01, or23);

    endmodule
\end{minted}


\subsubsection {A 16 to 1 Multiplexer}

\begin{minted} {verilog}
    module mux16(input logic [15:0] in, input logic [3:0] sl,
        output logic out, muxlast);

        wire sl00, sl01, sl02, sl03; // original select lines
        wire sl10, sl11, sl12, sl13; // inverted select lines

        assign sl00 = sl[0];
        assign sl01 = sl[1];
        assign sl02 = sl[2];
        assign sl03 = sl[3];

        nor #3 (sl10, sl00, sl00);
        nor #3 (sl11, sl01, sl01);
        nor #3 (sl12, sl02, sl02);
        nor #3 (sl13, sl03, sl03);

        wire nor00, nor01, nor02, nor03, nor04, nor05, nor06, nor07,
            nor08, nor09, nor10, nor11, nor12, nor13, nor14, nor15;

        nor5input nor500 (.in0(sl03), .in1(sl02), .in2(sl01), .in3(sl00), .in4(in[00]),
        .out(nor00));
        nor5input nor501 (.in0(sl03), .in1(sl02), .in2(sl01), .in3(sl10), .in4(in[01]),
        .out(nor01));
        nor5input nor502 (.in0(sl03), .in1(sl02), .in2(sl11), .in3(sl00), .in4(in[02]),
        .out(nor02));
        nor5input nor503 (.in0(sl03), .in1(sl02), .in2(sl11), .in3(sl10), .in4(in[03]),
        .out(nor03));
        nor5input nor504 (.in0(sl03), .in1(sl12), .in2(sl01), .in3(sl00), .in4(in[04]),
        .out(nor04));
        nor5input nor505 (.in0(sl03), .in1(sl12), .in2(sl01), .in3(sl10), .in4(in[05]),
        .out(nor05));
        nor5input nor506 (.in0(sl03), .in1(sl12), .in2(sl11), .in3(sl00), .in4(in[06]),
        .out(nor06));
        nor5input nor507 (.in0(sl03), .in1(sl12), .in2(sl11), .in3(sl10), .in4(in[07]),
        .out(nor07));
        nor5input nor508 (.in0(sl13), .in1(sl02), .in2(sl01), .in3(sl00), .in4(in[08]),
        .out(nor08));
        nor5input nor509 (.in0(sl13), .in1(sl02), .in2(sl01), .in3(sl10), .in4(in[09]),
        .out(nor09));
        nor5input nor510 (.in0(sl13), .in1(sl02), .in2(sl11), .in3(sl00), .in4(in[10]),
        .out(nor10));
        nor5input nor511 (.in0(sl13), .in1(sl02), .in2(sl11), .in3(sl10), .in4(in[11]),
        .out(nor11));
        nor5input nor512 (.in0(sl13), .in1(sl12), .in2(sl01), .in3(sl00), .in4(in[12]),
        .out(nor12));
        nor5input nor513 (.in0(sl13), .in1(sl12), .in2(sl01), .in3(sl10), .in4(in[13]),
        .out(nor13));
        nor5input nor514 (.in0(sl13), .in1(sl12), .in2(sl11), .in3(sl00), .in4(in[14]),
        .out(nor14));
        nor5input nor515 (.in0(sl13), .in1(sl12), .in2(sl11), .in3(sl10), .in4(in[15]),
        .out(nor15), .f4high(muxlast));

        wire fnor0, fnor1, fnor2, fnor3;
        wire for0, for1, for2, for3;

        nor #3 (for0, fnor0, fnor0);
        nor #3 (for1, fnor1, fnor1);
        nor #3 (for2, fnor2, fnor2);
        nor #3 (for3, fnor3, fnor3);

        nor4input nor400 (.in0(nor00), .in1(nor01), .in2(nor02), .in3(nor03), .out(fnor0));
        nor4input nor401 (.in0(nor04), .in1(nor05), .in2(nor06), .in3(nor07), .out(fnor1));
        nor4input nor402 (.in0(nor08), .in1(nor09), .in2(nor10), .in3(nor11), .out(fnor2));
        nor4input nor403 (.in0(nor12), .in1(nor13), .in2(nor14), .in3(nor15), .out(fnor3));

        nor4input nor404 (.in0(for0), .in1(for1), .in2(for2), .in3(for3), .out(out));

    endmodule
\end{minted}

\clearpage

\subsubsection {An Abstracted Schematic of A 16 to 1 Multiplexer}

\begin{figure} [!ht]
    \centering
    \resizebox{\textwidth}{!} {
        \begin{tikzpicture}

            \draw [thick, color = catppuccinfg]
            % \draw [thick, color = {rgb:red,100;green,50;blue,20}]
            % \draw [thick, color = \color{draculafg}]
            (0,0)
            \muxSixteen {mux} {out}

            \labelmuxSixteenFl {mux}
            {\texttt{FL 07}}
            {\texttt{FL 06}}
            {\texttt{FL 05}}
            {\texttt{FL 04}}
            {\texttt{FL 03}}
            {\texttt{FL 02}}
            {\texttt{FL 01}}
            {\texttt{FL 00}}

            \labelmuxSixteenSl {mux}
            {\texttt{SECOND LAYER}}

            (mux-in00) node [ocirc] {}
            (mux-in01) node [ocirc] {}
            (mux-in02) node [ocirc] {}
            (mux-in03) node [ocirc] {}
            (mux-in04) node [ocirc] {}
            (mux-in05) node [ocirc] {}
            (mux-in06) node [ocirc] {}
            (mux-in07) node [ocirc] {}
            (mux-in08) node [ocirc] {}
            (mux-in09) node [ocirc] {}
            (mux-in10) node [ocirc] {}
            (mux-in11) node [ocirc] {}
            (mux-in12) node [ocirc] {}
            (mux-in13) node [ocirc] {}
            (mux-in14) node [ocirc] {}
            (mux-in15) node [ocirc] {}

            (mux-in00) node [above = 4pt] {$In_{00}$}
            (mux-in01) node [above = 4pt] {$In_{01}$}
            (mux-in02) node [above = 4pt] {$In_{02}$}
            (mux-in03) node [above = 4pt] {$In_{03}$}
            (mux-in04) node [above = 4pt] {$In_{04}$}
            (mux-in05) node [above = 4pt] {$In_{05}$}
            (mux-in06) node [above = 4pt] {$In_{06}$}
            (mux-in07) node [above = 4pt] {$In_{07}$}
            (mux-in08) node [above = 4pt] {$In_{08}$}
            (mux-in09) node [above = 4pt] {$In_{09}$}
            (mux-in10) node [above = 4pt] {$In_{10}$}
            (mux-in11) node [above = 4pt] {$In_{11}$}
            (mux-in12) node [above = 4pt] {$In_{12}$}
            (mux-in13) node [above = 4pt] {$In_{13}$}
            (mux-in14) node [above = 4pt] {$In_{14}$}
            (mux-in15) node [above = 4pt] {$In_{15}$}

            (mux-sl0) node [ocirc] {}
            (mux-sl1) node [ocirc] {}
            (mux-sl2) node [ocirc] {}
            (mux-sl3) node [ocirc] {}

            (mux-sl0) node [left = 4pt] {$Sl_{0}$}
            (mux-sl1) node [left = 4pt] {$Sl_{1}$}
            (mux-sl2) node [left = 4pt] {$Sl_{2}$}
            (mux-sl3) node [left = 4pt] {$Sl_{3}$}

            (mux-out) node [ocirc] {}
            (mux-out) node [below = 4pt] {$Out$}

            (mux-muxlast) node [circleinv, fill = black, anchor = top] {}

            (mux-muxlast)
            -- (mux-muxlast|-mux-out)

            coordinate (muxlast)
            (muxlast) node [ocirc] {}
            (muxlast) node [below = 4pt] {$Muxlast$}

            ;

        \end{tikzpicture}
    }
    \caption{Schematic of a 16 to 1 MUX}
\end{figure}

\subsection {Edge Dectectors}

\subsubsection {Falling Edge Dectector}

\begin{minted} {verilog}
    module edge_detector(input logic clk,
        output logic feclk);

        /*
         * Falling edge detector
         * */

        wire notclk;

        nor #30 (notclk, clk, clk);
        nor #3 (feclk, notclk, clk);

    endmodule
\end{minted}

\subsubsection {Falling Edge Dectector with Inverted Output}

\begin{minted} {verilog}
    module edge_detector_inv(input logic clk,
        output logic notfeclk);

        /*
         * Falling edge detector,
         * with inverted output
         * */

        wire notclk;
        wire feclk;

        nor #30 (notclk, clk, clk);
        nor #3 (feclk, notclk, clk);
        nor #3 (notfeclk, feclk, feclk);

    endmodule
\end{minted}

\subsection {Logic Gates}

\subsubsection {XOR Gate}

\begin{minted} {verilog}
    module xor_gate(input logic a, b,
        output logic y);

        wire anorb, nor1, nor2, xnor1;

        nor #3 (anorb, a, b);
        nor #3 (nor1, anorb, a);
        nor #3 (nor2, anorb, b);
        nor #3 (xnor1, nor1, nor2);
        nor #3 (y, xnor1, xnor1);

    endmodule
\end{minted}

\subsubsection {AND Gate}

\begin{minted} {verilog}
    module and_gate(input logic a, b,
        output logic y);

        wire anot, bnot;

        nor #3 (anot, a, a);
        nor #3 (bnot, b, b);
        nor #3 (y, anot, bnot);

    endmodule
\end{minted}

\subsubsection {OR Gate}

\begin{minted} {verilog}
    module or_gate(input logic a, b,
        output logic y);

        wire anorb;

        nor #3 (anorb, a, b);
        nor #3 (y, anorb, anorb);

    endmodule
\end{minted}

\subsubsection {NOT Gate}

\begin{minted} {verilog}
    module not_gate(input logic a,
        output logic y);

        nor #3 (y, a, a);

    endmodule
\end{minted}

\subsubsection {NOR Gate}

\begin{minted} {verilog}
    module nor_gate(input logic a, b,
        output logic y);

        nor #3 (y, a, b);

    endmodule
\end{minted}

\subsubsection {NAND Gate}

\begin{minted} {verilog}
    module nand_gate(input logic a, b,
        output logic y);

        wire nota, notb;
        wire andab;

        nor #3 (nota, a, a);
        nor #3 (notb, b, b);
        nor #3 (andab, nota, notb);
        nor #3 (y, andab, andab);

    endmodule

\end{minted}

\subsection {Full Adder}

\begin{minted} {verilog}
    module full_adder(input logic a, b, cin,
        output logic sum, cout);

        wire anorb;
        wire xnor1, xnor2;
        wire xnor1nor1, xnor1nor2, xnor2nor1, xnor2nor2;
        wire xnor1norcin;

        // sum
        nor #3 (anorb, a, b);
        nor #3 (xnor1nor1, a, anorb);
        nor #3 (xnor1nor2, b, anorb);
        nor #3 (xnor1, xnor1nor1, xnor1nor2);

        nor #3 (xnor1norcin, xnor1, cin);
        nor #3 (xnor2nor1, xnor1, xnor1norcin);
        nor #3 (xnor2nor2, cin, xnor1norcin);
        nor #3 (sum, xnor2nor1, xnor2nor2);

        // cout
        nor #3 (cout, xnor1norcin, anorb);

    endmodule
\end{minted}

%%% \subsection{Flipflops}
%%%
%%% \newcommand\testing[1] {
%%%
%%% }
%%%
%%% \tikzset{external/force remake=true}
%%%
%%% \subsubsection {SR Latch}
%%%
%%% \begin{figure} [!ht]
%%%     \centering
%%%     \resizebox{\textwidth}{!} {
%%%         \begin{tikzpicture}
%%%             \draw
%%%             \testing {billie}
%%%             ;
%%%         \end{tikzpicture}
%%%     }
%%%     \caption{SR Latch}
%%% \end{figure}
%%%
%%% \tikzset{external/force remake=false}
%%%
\end{document}
